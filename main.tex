%%% Capítulos estão dentro da pasta /contents

\documentclass[tcc,numbers]{coppe}
\usepackage{amsmath,amssymb}
\usepackage{hyperref}

\makelosymbols
\makeloabbreviations

\begin{document}
  \title{T\'itulo da Tese}
  \foreigntitle{Thesis Title}
  \author{Miguel}{Fernandes de Sousa}
  \advisor{Prof.}{Luiz Wagner}{Pereira Biscainho}{D.Sc.}
  \advisor{Prof.}{Nome do Segundo Orientador}{Sobrenome}{Ph.D.}
  \advisor{Prof.}{Nome do Terceiro Orientador}{Sobrenome}{D.Sc.}

  \examiner{Prof.}{Nome do Primeiro Examinador Sobrenome}{D.Sc.}
  \examiner{Prof.}{Nome do Segundo Examinador Sobrenome}{Ph.D.}
  \examiner{Prof.}{Nome do Terceiro Examinador Sobrenome}{D.Sc.}
  \examiner{Prof.}{Nome do Quarto Examinador Sobrenome}{Ph.D.}
  \examiner{Prof.}{Nome do Quinto Examinador Sobrenome}{Ph.D.}
  \department{ELE}
  \date{07}{2020}

  \keyword{Espectrograma de Modulação}
  \keyword{Segunda palavra-chave}
  \keyword{Terceira palavra-chave}

  \maketitle

  \frontmatter
  \dedication{A algu\'em cujo valor \'e digno desta dedicat\'oria.}

  \chapter*{Agradecimentos}

  Gostaria de agradecer a todos.

  \begin{abstract}

  Apresenta-se, nesta tese, ...

  \end{abstract}

  \begin{foreignabstract}

  In this work, we present ...

  \end{foreignabstract}

  \tableofcontents
  \listoffigures
  \listoftables
  \printlosymbols
  \printloabbreviations

  \mainmatter
  %%________________________
  %% CHAPTER 01
  \chapter{Introdu{\c c}\~ao}
\section{Percepção de periodicidade}
A audição humana é capaz de conferir qualidades subjetivas aos sons, associadas
a parametros físicos nele presentes. Intensidade, timbre, duração e direção percebidas
são exemplos de qualidades subjetivas, enquanto que pressão, frequência,
espectro e envelope são exemplos de parâmetros físicos.\cite{rossing2002}

A audição também é capaz de distinguir sons periódicos de sons "aperiódicos", ou seja,
 sons sem periodicidade definida.

A percepção de periodicidade no sistema auditivo possui qualidades distintas,
 dependendo da frequência do som audível. Frequências abaixo de
20Hz podem ser percebidas ritmicamente, enquanto que frequências audíveis acima de
 20Hz podem ser percebidas como Pitch.

Pitch é a percepção associada à tonalidade do som, na qual o ouvinte é capaz
de distinguir sons agudos e graves, e de ordená-los conforme sua tonalidade.
 Em geral, sons com pitch percebido podem ser representados como a mistura de uma
oscilação no tempo com frequência fundamental, agregada às suas parciais, ou
 harmônicos. Sons ``aperiódicos'' não possuem pitch definido, uma vez que as frequências das
 componentes de sua mistura não são distribuídas harmonicamente. \cite{langner1992} \cite{angus2009} 

 Um som pode ser composto por uma forma de onda periódica, que
 repete-se em função de sua frequência fundamental f, ou período T, sendo essa 
 frequência fundamental percebida como pitch. Essa definição distingue-se de um
 trem de pulsos com determinada taxa de repetição, que não é diretamente
 associada à sua frequência fundamental. Apesar disso, o trem de pulsos pode ser
considerado um som periódico. Os dois casos acima demonstram que o sistema
auditivo realiza tanto uma análise em frequência quanto uma análise no tempo
ao perceber sons. \cite{rossing2002} \symbl{T}{Período} \symbl{f}{Frequência}


A percepção de periodicidade do som também pode ser conferida ao envelope,
definido como a variação no tempo da amplitude ou energia de uma vibração.
Caso o envelope oscile com uma determinada frequência, gera-se uma onda
modulada em amplitude.
Essa modulação pode ser percebida como uma segunda periodicidade,
além da frequência fundamental dessa mesma onda. Dessa forma, vale destacar
que há múltiplas dimensões temporais no estimulo acústico, e pode-se distinguir
a "estrutura fina", associada ao conteúdo espectral, de seu contorno caracterizado
pelo envelope modulado em frequência, ambos compondo a forma de onda no
domínio do tempo. \cite{joris2004}

\citet{langner1992} distingue modulações lentas, abaixo de 20Hz, de
modulações rápidas, entre 20Hz e 1000Hz. Essa distinção se dá pela percepção
distinta nos dois casos: modulações lentas estão associadas à percepção de
ritmo e também à taxa de sequência na construção de palavras, entre 3Hz e 4Hz,
enquanto que modulações rápidas estão associadas a sensação desagradável
de "irregularidade"(roughness) no som. 

\citet{zwicker2013} Apresentam detalhes sobre os limiares de percepção da
Modulação de Amplitude (AM) em função das frequências e intensidades, \citet{joris2004} e
\citet{langner1992} avaliam a resposta à modulação de cada componente do 
sistema auditivo e nervoso, indo além do escopo do
presente trabalho. Modulações temporais são estímulos que auxiliam na detecção,
discriminação, identificação e localização de fontes sonoras,
e esse papel amplo reflete-se em propriedades fisiológicas particulares a cada
componente do sistema auditivo. \abbrev{AM}{Modulação de Amplitude}

Falta falar: brevemente, como o ouvido percebe a modulação ; brevemente, como modulação,
 envelope e formantes da fala se correlacionam; percepção de timbre de instrumentos
  associado ao envelope de cada frequência; inserir imagem.

\section{Descrição do projeto}
\subsection{Objetivos}
\subsection{Organização do Trabalho}
\subsection{Materiais Utilizados}

  %%________________________

    %%________________________
  %% CHAPTER 02
  \chapter{Fundamentação Teórica}
\section{Correspondência entre inteligibilidade e modulação}
% Nervo Auditivo:
% - resolução em frequência e resposta da modulação
% - diversidade de resposta
% - caracteristica passa baixa das fibras do nervo auditivo
% - faixa dinâmica

% Nucleo coclear:
% - resolução em frequência e resposta da modulação
% - diversidade de resposta
% - caracteristica passa baixa das fibras do nervo auditivo
% - faixa dinâmica

% A audição humana pode ser aproximada por um filtro
%  passa-baixas.\cite{langner1992}.

% Modulações abaixo de 20Hz são percebidas ritmicamente. Modulações predominantes
% entre 3 e 4Hz coencidem com a taxa que pronunciamos palavras. \cite{langner1992}.

% Modulações entre 10Hz e 200Hz são percebidas como desagradáveis. \cite{langner1992}.

% \cite{langner1992} define a percepção de periodicidade do envelope como 
% "periodicity pitch". 

% MTF: modulation transfer function.

% A percepção de pitch trata da frequência percebida. Dois sinais complexos
% compostos por diversas componentes frequênciais distintas podem ter o mesmo
%  pitch. Ainda, um sinal, por mais que tenha determinado pitch, pode não conter essa
%  frequência em sua composição. A percepção de pitch funciona por mais que não esteja presente a 
%  frequência fundamental de mesmo valor daquele pitch. \cite{langner1992}

%  Nervo auditivo. Núcleo Coclear. \cite{langner1992}.

%  This indicates that the auditory system tends to separate information about
%   the envelope and the temporal fine structure of a signal as
%    a first step of temporal analysis of sounds.


% Efeito do envelope na percepção de fala:

% \cite{drullman1994} O sinal de fala é caracterizado por um espectro de
%  frequências variante no tempo. Essas variações são responsáveis pela
%  identificação de fonemas, sílabas, palavras e frases.

%  Em todas as bandas, as frequências mais importantes são as presentes entre
%  3 e 4 Hz, refletindo a taxa silábica da fala. É possível encontrar componentes
%  na faixa entre 15-20hz.

%  A sensibilidade para modulação tem caracteristica de um filtro passa-baixa,
%  com queda de 6dB entre 25Hz e 100Hz.

%  Posso colocar um gráfico aqui. Como que se mede esse gráfico?

%  MI: Modulation Index.
%  SRT: Speech-Reception Threshold.
%  STI: Speech-Transmition Index.

%  Em áudio reverberante, a atenuação da modulação diminiu a inteligibilidade de 
%  frases. \cite{drullman1994}. Compressão, por subbandas ou diretamente no sinal,
%  afeta drasticamente a inteligibilidade.

%  Essa informação é útil para síntese de voz? Dado que eu quero sintetizar
%  uma voz com inteligibilidade alta. E para análise do som, em aparelhos para
%  deficientes auditivos?

%  O efeito de borramento é reproduzido em áudio reverberante, ao convoluir
%  um sinal com uma exponencial que decai.

%  Pode ser medida em termos de matriz de confusão.

%  SIQ (sentence intelligibility in quiet).

%  Qual a tradução de smearing? Achatamento, borramento, espalhamento?

%  Detecção de envelope com transformada de Hilbert.
 
% Consoantes são mais afetadas por borramento temporal do que vogais.
\section{Cicloestacionariedade de Segunda Ordem e de Ordem Superior}

\section{Correspondência entre inteligibilidade e modulação}
\section{Speech Enhancement}
˜\newpage 
˜\newpage
\section{STFT}
\newpage 
˜\newpage 
\section{Framework AMS}
\newpage 
˜\newpage 
\section{Espectrograma de modulação}
\newpage 
˜\newpage 
˜\newpage
  %%________________________


  %%________________________
  %% TEMPLATE HELPER
  % \input{documentation/template.tex}
  %%________________________




  \backmatter
  \bibliographystyle{coppe-unsrt}
  \bibliography{main}

  \appendix
  \chapter{Algumas Demonstra{\c c}\~oes}
\end{document}
%% 
%%
%% End of file `example.tex'.
