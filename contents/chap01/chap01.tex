\chapter{Introdu{\c c}\~ao}
\section{Percepção de periodicidade}
A audição humana é capaz de conferir qualidades subjetivas aos sons, associadas
a parametros físicos presentes no som. Intensidade, timbre, duração e direção percebidas
são exemplos de qualidades subjetivas, enquanto que pressão, frequência,
espectro e envelope são exemplos de parâmetros físicos.\cite{rossing2002}

A audição também é capaz de distinguir sons periódicos de sons "aperiódicos", ou seja,
 sons sem periodicidade definida.

A percepção de periodicidade no sistema auditivo possui qualidades distintas,
 dependendo da frequência do som audível. Frequências abaixo de
20Hz podem ser percebidas ritmicamente, enquanto que frequências audíveis acima de
 20Hz podem ser percebidas como Pitch. [lack reference]

Pitch é a percepção associada à tonalidade do som, na qual o ouvinte é capaz
de distinguir sons agudos e graves, e de ordená-los conforme sua tonalidade.
 \cite{angus2009}

Em geral, sons com pitch percebido podem ser representados como a mistura de uma
oscilação no tempo com frequência fundamental, agregada às suas parciais, ou
 harmônicos.  Sons "aperiódicos" não possuem pitch definido, uma vez que as frequências das
 componentes de sua mistura não são distribuidas harmonicamente. \cite{langner1992}

 Um som periódico pode ser composto por uma forma de onda periódica, que
 repete-se em função de sua frequência fundamental f, ou período T, sendo essa 
 frequência fundamental percebida como pitch. Essa definição distingue-se de um
 trem de pulsos com taxa de repetição baixa, não é diretamente associada à sua frequência fundamental.
 [falo que abaixo de determinada frerquencia, trem de pulsos nao gera percepcao de pitch?]
 Apesar disso, o trem de pulsos pode ser considerado um som
 periódico. Os dois casos acima demonstram que o sistema auditivo realiza tanto uma análise
 em frequência quanto uma análise no tempo ao perceber sons. \citet{Rossing}

A percepção de periodicidade do som também pode ser conferida ao envelope,
definido como a variação no tempo da amplitude ou energia de uma vibração.
Caso o envelope oscile com uma determinada frequência, gera-se uma onda
modulada em aplitude.
Essa modulação pode ser percebida como uma segunda periodicidade,
além da frequência fundamental dessa mesma onda. Dessa forma, vale destacar
que há multiplas dimensões temporais no estimulo acústico, e pode-se distinguir
a "estrutura fina", associada ao conteúdo espectral, de seu contorno caracterizado
pelo envelope modulado em frequência, ambos compondo a forma de onda no
domínio do tempo. \cite{joris2004}

Langner \citet{langner1992} distingue modulações lentas, abaixo de 20Hz, de
modulações rápidas, entre 20Hz e 1000Hz. Essa distinção se dá pela percepção
distinta nos dois casos: modulações lentas estão associadas à percepção de
ritmo e à taxa de sequência na construção de palavras, entre 3Hz e 4Hz, enquanto
que modulações rápidas estão associadas a sensação desagradável
de "irregularidade"(roughness) no som. 

\citet{zwicker2013} Apresenta detalhes sobre os limiares de percepção do
AM em função das frequências e intensidades, \citet{joris2004} e
\citet(langner1992) avaliam a resposta à modulação de cada componente do 
sistena auditivo e nervoso, indo além do escopo do
presente trabalho. Modulações temporais são estimulos que auxiliam na detecção,
discriminação, identificação e localização de fontes sonoras,
e esse papel amplo reflete-se em propriedades fisiológicas particulres a cada
componente do sistema auditivo.

temporal modulations are a critical stimulus attribute that assists us in the detection, discrimination, identifi- cation, parsing, and localization of acoustic sources and that this wide-ranging role is reflected in dedicated physiological properties at different anatomical levels.

Falta falar:
 - Onde o cérebro processa cada coisa
 - como modulação, envelope e formantes da fala se correlacionam


\cite{rossing2002}


\section{Resumo}
˜\newpage
\section{Motivação}
˜\newpage
\newpage˜
\section{Revisão Bibliográfica}
\section{Descrição do projeto}
\subsection{Objetivos}
\subsection{Organização do Trabalho}
\subsection{Materiais Utilizados}