\section{Correspondência entre inteligibilidade e modulação}
% Nervo Auditivo:
% - resolução em frequência e resposta da modulação
% - diversidade de resposta
% - caracteristica passa baixa das fibras do nervo auditivo
% - faixa dinâmica

% Nucleo coclear:
% - resolução em frequência e resposta da modulação
% - diversidade de resposta
% - caracteristica passa baixa das fibras do nervo auditivo
% - faixa dinâmica

% A audição humana pode ser aproximada por um filtro
%  passa-baixas.\cite{langner1992}.

% Modulações abaixo de 20Hz são percebidas ritmicamente. Modulações predominantes
% entre 3 e 4Hz coencidem com a taxa que pronunciamos palavras. \cite{langner1992}.

% Modulações entre 10Hz e 200Hz são percebidas como desagradáveis. \cite{langner1992}.

% \cite{langner1992} define a percepção de periodicidade do envelope como 
% "periodicity pitch". 

% MTF: modulation transfer function.

% A percepção de pitch trata da frequência percebida. Dois sinais complexos
% compostos por diversas componentes frequênciais distintas podem ter o mesmo
%  pitch. Ainda, um sinal, por mais que tenha determinado pitch, pode não conter essa
%  frequência em sua composição. A percepção de pitch funciona por mais que não esteja presente a 
%  frequência fundamental de mesmo valor daquele pitch. \cite{langner1992}

%  Nervo auditivo. Núcleo Coclear. \cite{langner1992}.

%  This indicates that the auditory system tends to separate information about
%   the envelope and the temporal fine structure of a signal as
%    a first step of temporal analysis of sounds.


% Efeito do envelope na percepção de fala:

% \cite{drullman1994} O sinal de fala é caracterizado por um espectro de
%  frequências variante no tempo. Essas variações são responsáveis pela
%  identificação de fonemas, sílabas, palavras e frases.

%  Em todas as bandas, as frequências mais importantes são as presentes entre
%  3 e 4 Hz, refletindo a taxa silábica da fala. É possível encontrar componentes
%  na faixa entre 15-20hz.

%  A sensibilidade para modulação tem caracteristica de um filtro passa-baixa,
%  com queda de 6dB entre 25Hz e 100Hz.

%  Posso colocar um gráfico aqui. Como que se mede esse gráfico?

%  MI: Modulation Index.
%  SRT: Speech-Reception Threshold.
%  STI: Speech-Transmition Index.

%  Em áudio reverberante, a atenuação da modulação diminiu a inteligibilidade de 
%  frases. \cite{drullman1994}. Compressão, por subbandas ou diretamente no sinal,
%  afeta drasticamente a inteligibilidade.

%  Essa informação é útil para síntese de voz? Dado que eu quero sintetizar
%  uma voz com inteligibilidade alta. E para análise do som, em aparelhos para
%  deficientes auditivos?

%  O efeito de borramento é reproduzido em áudio reverberante, ao convoluir
%  um sinal com uma exponencial que decai.

%  Pode ser medida em termos de matriz de confusão.

%  SIQ (sentence intelligibility in quiet).

%  Qual a tradução de smearing? Achatamento, borramento, espalhamento?

%  Detecção de envelope com transformada de Hilbert.
 
% Consoantes são mais afetadas por borramento temporal do que vogais.