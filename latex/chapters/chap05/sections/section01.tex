\section{Avaliação dos Resultados}
O presente trabalho demonstra que é possível utilizar técnicas de SBRS para
recomendação de usuários e indivíduos a grupos de participantes, o que é um
domínio novo na literatura de SBRS. Foi demonstrado que há modelos de base de
comparação eficazes e competitivos para a base de dados apresentada. Modelos
mais sofisticados, como os baseados em redes neurais, podem ser utilizados para
melhorar os resultados obtidos, ao custo de maior complexidade computacional,
tempo de treinamento e otimização, e maior necessidade de dados de treinamento.

Foi demonstrado que modelos \textit{session-aware}, quando comparados a seus
equivalentes \textit{session-based}, apresentam incremento de desempenho. Além
disso, alguns modelos \textit{session-aware} superam os modelos
\textit{session-based} de maior desempenho. Também foi demonstrado que a
abordagem \textit{last-item} obteve resultados equivalentes à abordagem
\textit{next-item}, por mais que os itens na abordagem \textit{last-item} sejam
necessariamente inéditos.

Ao comparar os resultados obtidos com os de outros trabalhos, observamos que há
consonância na ordem de grandeza dos resultados, indicando que a base de dados
criada para esse trabalho é adequada para o estudo de SBRS, apesar das
limitações de tamanho e desbalanceamento em razão do crescimento do aplicativo
Indaband. A tendência é que essas duas questões sejam mitigadas com o crescimento da base
de usuários.