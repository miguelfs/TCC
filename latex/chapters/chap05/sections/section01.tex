\section{Avaliação dos Resultados}

As medidas de avaliação obtidas com a base do presente trabalho demonstram que,
para as características específicas da base, incluindo escopo, tamanho e
quantidade média de itens por sessão, os modelos de SBRS apresentam resultados
satisfatórios que justificam seu uso em um ambiente de produção do aplicativo
Indaband. Além disso, os modelos \textit{session-aware} apresentaram os melhores
resultados entre todos os modelos avaliados, indicando melhora significativa ao
incluir o metadado do usuário dono da sessão. Entre os modelos
\textit{session-based}, destacaram-se modelos baseados em regras de sequência,
vizinhos mais próximos, árvores de contexto, smf e GNN.  Os resultados positivos para esses modelos
condizem com medidas de avaliação em publicações mais
recentes~\cite{shehzad2024performance}.

Por mais que a base INDABAND contenha menos sessões do que as demais bases da
literatura, os valores para as métricas como HR@5, HR@10, MRR@5 e MRR@10 são
compatíveis com os de outras bases maiores, como RSC15, ZALANDO, CLEF e
RETAILROCKET. Um cenário possível, com o crescimento da base de usuários ao
longo do tempo, é o das métricas de avaliação obterem valores maiores, em
particular para os modelos baseados em redes neurais. Na FIGURA
\ref{fig:next-item-single-123}, é possível observar a tendência de melhora no
HR@5 para cada \textit{split}, cada um contendo mais sessões do que
\textit{split} anterior, como destacado na TABELA \ref{tab:windowed_data}. A
tendência é que o desbalanceamento de sessões ao longo do tempo seja mitigado
com o crescimento da base de usuários.

O presente trabalho demonstra que é possível modelar e utilizar técnicas de SBRS
para recomendação de usuários e indivíduos a grupos de participantes, o que é um
domínio novo na literatura de SBRS. Modelos mais sofisticados, como os baseados
em redes neurais, podem ser utilizados para melhorar os resultados obtidos, ao
custo de maior complexidade computacional, tempo de treinamento e otimização.
Apesar disso, modelos mais simples, como os baseados em regras de sequência ou
em vizinhos mais próximos, obtém resultados satisfatórios, superiores à maioria
modelos de redes neurais do comparativo.

A abordagem \textit{last-item} obteve resultados equivalentes à abordagem
\textit{next-item}, por mais que os itens avaliados na abordagem
\textit{last-item} sejam necessariamente inéditos para suas respectivas sessões.
Esse resultado indica que, mesmo em um cenário de avaliação mais exigente e mais
próximo do que a aplicação exige, os modelos SBRS entregam resultados
satisfatórios que justificam sua futura implementação na aplicação.
