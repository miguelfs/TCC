\section{Trabalhos Futuros}
Conforme a base de dados cresça, será possível comparará-la com as de outros
domínios de forma mais igualitária, utilizando os mesmos valores de suporte
mínimo ou de tamanho das janelas. Apesar dessas questões não terem comprometido
a realização dos experimentos, sobretudo nos comparativos entre modelos para
o mesmo conjunto de dados, o ideal é que novos pesquisadores não tenham que
se preocupar com divergências sutis nas especificações de cada banco.

No contexto da aplicação, a implementação do sistema \textit{session-based} ou
\textit{session-aware} é um próximo passo que permitirá avaliação prática dos
modelos propostos, inclusive com avaliação subjetiva ou de satisfação dos
usuários. Outros domínios na própria aplicação podem ser avaliados, a exemplo da
recomendação de conteúdo no \textit{Feed}, domínio pouco explorado na literatura
para esse tipo de recomendador.

Por mais que o presente trabalho indique quais modelos apresentaram
melhores medidas de avaliação em detrimento de modelos com resultados
inferiores, o presente trabalho não apresenta uma investigação sobre
os motivos pelos quais um modelo obteve melhores resultados do que
outro, ou por qual motivo alguns modelos obtiveram resultados
inferiores, apesar de se destacarem em comparativos com outras bases. Essa
investigação também pode ser um próximo passo para a compreensão do
comportamento dos modelos propostos.