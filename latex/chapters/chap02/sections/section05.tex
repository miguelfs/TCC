\section{Demodulação Incoerente e Coerente}
A demodulação, no contexto de telecomunicações, consiste em recuperar a parcela
original em banda base de um sinal modulado, transladando seu espectro para sua
posição original~\cite{lathi2017}. Os métodos disponíveis na bibliografia de
filtragem de modulação são separados em demodulações incoerentes, que estimam a
moduladora a partir de operações de módulo, e em demodulações coerentes, que
estimam a portadora a partir da frequência instantânea do sinal
\cite{clark2009time}.

Uma demodulação incoerente é a detecção de envoltória feita a partir da
transformada de Hilbert. A transformada de Hilbert é capaz de gerar o sinal em
quadratura, deslocando sua fase em $\pi/2$ radianos. Sua definição, no domínio
do tempo contínuo e seu
espectro~\cite{haykin2014digital}\cite{goulart2017efeitos}, são dados por:
\symbl{$\mathcal{H}\{\cdot\}$}{operador para transformada de Hilbert}
\symbl{$\ast$}{operador para convolução}
\symbl{$\hat{x}(t)$}{sinal em quadratura}
\begin{equation}
    \mathcal{H}\{x(t)\} = x(t) \ast \frac{1}{\pi t} = \frac{1}{\pi} \int_{-\infty}^{+\infty} \frac{x(\tau)}{t - \tau}  \,dx 
\end{equation}
\begin{equation}
    \hat{x}(t) = \mathcal{H}\{x(t)\}
\end{equation}
\begin{equation}
    \mathrm{sgn}(\omega) = \begin{cases}
        1,& \omega < 0\\
        0, & \omega = 0\\
        -1, & \omega < 0
        \end{cases}
\end{equation}
\begin{equation}
    \hat{X}(\omega) = -j\,\mathrm{sgn}(\omega)X(\omega) 
\end{equation}
em que $\mathcal{H}\{\cdot\}$ é o operador para a transformada de Hilbert,
$x(t)$ é um sinal de valores reais, $\ast$ é o operador para convolução,
$\hat{x}(t)$ é o sinal em quadratura, $\mathrm{sgn}(\omega)$ é a função sinal e
$\hat{X}(\omega)$ é o espectro do sinal em quadratura.

O método consiste em decompor um sinal real em partes moduladora e portadora a
partir de um sinal analítico, obtido pela soma do sinal em questão com sua
transformada de Hilbert:\todo{hilbert discreto?} \symbl{$x_{+}(t)$}{sinal
analítico no domínio do tempo}
\begin{equation} \label{eqn:x_analitico}
      x_{+}(t) = x(t) + \mathrm{j}\mathcal{H}\{x(t)\}
\end{equation}
\begin{equation}
    |x_{+}(t)| = \sqrt{x^2(t) + \hat{x}^2(t)}
\end{equation}
\begin{equation}
    \phi(t) = \arctan\left( \frac{\hat{x}(t)}{x(t)} \right)
\end{equation}
\begin{equation}
    X_{+}(\omega) = X(\omega) + \mathrm{sgn}(\omega)\hat{X}(\omega)
\end{equation}
\begin{equation}
    X_{+}(\omega) = \begin{cases}
        2X(\omega),& \omega < 0\\
        X(0), & \omega = 0\\
        0, & \omega < 0
        \end{cases}
\end{equation}
em que $x_{+}(t)$, $|x_{+}(t)|$ e $\phi(t)$ são, respectivamente, o sinal
analítico, seu módulo e fase, $X_{+}(\omega)$ é o espectro do sinal analítico.

Em seguida, decompõem-se a moduladora e a portadora a partir do módulo e fase do
sinal analítico:
\symbl{$a(t)$}{envoltória no domínio do tempo contínuo}
\begin{equation}  \label{eqn:a_k}
      a(t) = |x_{+}(t)|
\end{equation}
\begin{equation}  \label{eqn:c_k}
      c(t) = \cos \phi(t) = \mathrm{e}^{\mathrm{j} \phi(t)}
\end{equation}
\symbl{$\phi(t)$}{fase do sinal analítico}
em que, $a(t)$ é a envoltória, $c(t)$ é a portadora. Note que, ao estimar a
envoltória a partir do módulo do sinal analítico, seus valores serão reais
não-negativos.\todo{discreto e sub banda}

No contexto de sub-bandas, em que as envoltórias e portadoras são estimadas para
cada sub-banda $k$, tal como na etapa de demodulação na filtragem de modulação,
as equações \ref{eqn:x_analitico}, \ref{eqn:a_k} e \ref{eqn:c_k} são aplicadas
em cada sub-banda:
\symbl{$m_k(t)$}{moduladora da k-ésima sub-banda, no domínio do tempo contínuo}
\symbl{$c_k(t)$}{portadora da k-ésima sub-banda, no domínio do tempo contínuo}
\symbl{$a_k(t)$}{envoltória da k-ésima sub-banda, no domínio do tempo contínuo}
\begin{equation}
    x_{k,+}(t) = x_{k}(t) + \mathrm{j}\mathcal{H}\{x_k(t)\}
\end{equation}
\begin{equation}
    m_k(t) = a_k(t) = |x_{k,+}(t)|
\end{equation}
\begin{equation}
    c_k(t) = \cos \phi_k(t) = \mathrm{e}^{\mathrm{j} \phi_k(t)}
\end{equation}

O método acima possui limitações importantes~\cite{schimmel2007}. A largura de
banda da portadora estimada espalha-se além dos limites da banda original do
sinal. Esse aspecto inviabiliza a etapa de reconstrução em bancos de filtros,
como na filtragem de modulação. Para exemplificar essa limitação, considere o
sinal complexo $y(t)$:
\begin{equation}
    y(t) = \cos(\omega_m t) e^{j\omega_c t}, \quad \omega_c > \omega_m 
\end{equation}
em que $\omega_c$ e $\omega_m$ são as frequências da portadora e moduladora. O
módulo ao quadrado do sinal é dado por
\begin{equation}
    |y(t)|^2  = y(t)y^*(t)
\end{equation}
\begin{equation}
    |y(t)|^2 = \cos^2(\omega_m t)
\end{equation}
\begin{equation}
    |y(t)|^2 = \frac{1}{2}\cos(2 \omega_m t) + \frac{1}{2}
\end{equation}
cujo espectro é limitado em banda. Por sua vez, o módulo de $y(t)$ é dado por:
\symbl{$\omega_m$}{frequência angular de modulação}
\symbl{$\omega_c$}{frequência angular da portadora}
\begin{equation}
   |y(t)| = |e^{j\omega_c t}\cos(\omega_m t)|
\end{equation}
\begin{equation}
    |y(t)| = |\cos(\omega_m t)|
\end{equation}
que contém descontinuidades em:
\begin{equation}
t = \frac{\pi}{\omega_m} (\frac{1}{2}+ \mathbf{k}), \quad \mathbf{k} \in \mathbb{Z} .
\end{equation}
Para representar essas descontinuidades, são necessárias infinitas
frequências.\todo{calcular limite?} O espectro do módulo de $y(t)$ é dado pela
equação~\ref{eq_spec_mod_tone}:
\begin{equation}
     m(t) = |y(t)|
\end{equation}
\begin{equation}
    M(\omega) = \mathcal{F}\{|y(t)|\}
\end{equation}
\begin{equation} \label{eq_spec_mod_tone}
    M(\omega) = \sum_{\mathbf{k} = 1}^{\infty}\frac{4}{-(-1)^\mathbf{k} ((2\mathbf{k})^2) - 1)} \delta(\omega - 2 \mathbf{k} \omega_m)   
\end{equation}
\symbl{$\mathbb{Z}$}{Conjunto dos números inteiros}
% \symbl{$\mathbf{k}$}{índice do somatório}
\todo{eq acima apêndice}
em que $\mathcal{F\{\cdot\}}$ é o operador para a transformada de Fourier.
\citet{cohen1999ambiguity} demonstra que a representação de módulo e fase não é
única, de forma que um sinal complexo possui múltiplos pares módulo-fase que o
representam. Tal como na demodulação incoerente, haverão descontinuidades nessas
representações caso sua amplitude não seja positiva definida.\todo{adicionar
grafico exemplificando}

Como alternativa, há a detecção coerente de
portadora~\cite{atlas2005,clark2009time,clark2009sum}, que visa à detecção de
uma portadora com banda estreita.

Para compreender as diferenças entre demodulação coerente e incoerente,
representamos a equação \ref{eq:signal_eq} na forma polar \cite{schimmel2007}:

\begin{equation}  \label{eq:polar}
    a_x(t) e^{j \phi_x(t)} = \left[ a_m(t) e^{j \phi_m(t)} \right] \left[ a_c(t) e^{j \phi_c(t)}\right]
\end{equation}
nessa representação, a envoltória $a_x(t)$ e fase $\phi_x(t)$ do sinal são dados
pela composição das envoltórias e fases da portadora e moduladora:
\begin{equation}
    a_x(t) = a_m(t) a_c(t)
\end{equation}
\begin{equation}
    \phi_x(t) = \phi_m(t) + \phi_c(t)    
\end{equation}
Como discutido por \citet{cohen1999ambiguity}, há múltiplas soluções\todo{falar
da fase} para a equação \ref{eq:polar}. A detecção incoerente assume que:
\begin{equation}
    \phi_m(t) = 0
\end{equation}
\begin{equation}
    a_c(t) = 1
\end{equation}
\begin{equation}
    a_m(t) = a_x(t)
\end{equation}
\begin{equation}
    \phi_c(t) = \phi_x(t)
\end{equation}
Diferentemente, a detecção coerente considera que a parcela moduladora assuma
valores complexos, de forma que $\phi_m(t)$ tenha valores não-nulos. Para
resolver a ambiguidade na decomposição das fases de $\phi_m(t)$ e $\phi_c(t)$,
impõe-se a restrição de que $\phi_c(t)$ recebe apenas componentes de fase com
lenta variação, em relação à fase do sinal original.
% O que preserva a largura de banda.

A fase $\phi_c(t)$ da portadora é estimada a partir da frequência instantânea do
sinal $x(t)$, na qual aplica-se um filtro passa-baixas $h_{lp}(t)$, integrando-a
no tempo:
\begin{equation}
    \alpha_c(t) = \frac{d\phi_x(t)}{dt} * h_{lp}(t)
\end{equation}
\begin{equation}\label{eqn:carrier_integral}
    \phi_c(t) = \int_{0}^{t}\alpha_c(t)\,d\tau 
\end{equation}

Em que $\alpha_c(t)$ é a frequência instantânea filtrada. Note que a derivada de
$\phi_x(t)$ no tempo expressa a frequência instantânea de $x(t)$. Finalmente, a
parte moduladora é obtida ao descontar do sinal original a estimativa de sua
parcela portadora:
\begin{equation} \label{eqn:carrier}
    c(t) = e^{j\phi_c(t)}
\end{equation}
\begin{equation}
    m(t) = \frac{x(t)}{c(t)} = x(t) \cdot  c^*(t)
\end{equation}
Substituindo a equação:
\begin{equation}\label{eqn:demod_coh}
    m(t) = \frac{x(t)}{e^{j\phi_c(t)}} = x(t)e^{-j\phi_c(t)}.
\end{equation}
Existem formas distintas de realizar a demodulação coerente. Em comum,
compartilham a estimativa das componentes a partir da fase do sinal, preservando
a largura de banda final.

A demodulação coerente por centro de gravidade espectral \cite{clark2009time}
\cite{clark2009sum} \cite{toolbox2010} estima a portadora da equação
\ref{eqn:carrier} a partir da frequência que corresponde ao centróide da
distribuição, denotada por $\omega_0$:
\begin{equation} \label{eqn:cog}
    \omega_0 = \frac{\int_{-\infty}^{+\infty}\omega S_{xx}(\omega)\,d\omega}{\int_{-\infty}^{+\infty}S_{xx}(\omega)\,d\omega} 
\end{equation}
\begin{equation} \label{eqn:cog_carrier}
    c(t) = e^{j\omega_0 t}
\end{equation}
em que $S_{xx}(\omega)$ é a densidade espectral de potência do sinal original
$x(t)$. Note que a equação \ref{eqn:cog} aplica-se para sinais estacionários,
enquanto que a equação \ref{eqn:cog_carrier} equivale à equação
\ref{eqn:carrier_integral}, com a fase proporcional ao tempo.
% fase proporcional ao tempo, apenas para w_0
% 

No domínio discreto, para sinais não-estacionários, a frequência instantânea
para o $l$-ésimo \textit{frame} é estimada a partir da STFT. No contexto de
filtragem por sub-bandas:

\begin{equation}
    f_k[l] = \frac{\sum_{\mathcal{K}=0}^{M-1} r[l] |X_k[l,\mathcal{K}]|^2}{|X_k[l,\mathcal{K}]|^2}
\end{equation}
em que $f_k[n]$ é a frequência instantânea da $k$-ésima sub-banda, $\mathcal{K}$
é o índice do \textit{bin} e $M$ é metade do tamanho do \textit{frame} $N$, sendo
$   $ par, em amostras.

Uma segunda demodulação coerente, também presente na bibliografia, estima a
portadora a partir da detecção de frequência fundamental $F_0$ do sinal
original, de forma análoga à equação \ref{eqn:carrier_integral}, no domínio do
tempo discreto:
\begin{equation}
    \phi_0[n] = \sum_{p=0}^n F_0[n]
\end{equation}
Em que $F_0$ é o \textit{pitch} estimado para $x[n]$. Esse método, chamado
demodulação coerente harmônica, considera que cada sub-banda está relacionada a
um harmônico múltiplo da frequência fundamental, tal que a portadora seja obtida
por:
\begin{equation}
    c_k[n] = e^{jk\phi_0[n]}
\end{equation}
\begin{equation}
    m_k[n] = \sum^n_{p=0} h_{lp}[n-p] \cdot x[p] \cdot c^*_k[p].
\end{equation}
Tal que a parcela moduladora seja obtida ao aplicar filtragem passa-baixas no
sinal original, previamente deduzido de sua portadora.