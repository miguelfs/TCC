\section{Sistemas de Recomendação \textit{Session-Based} e \textit{Session-Aware}}

Os sistemas de recomendação baseados em sessão (SBRS, do inglês
 \textit{session-based recommendation systems}) são aplicáveis quando os dados
 de entrada do modelo consistem em uma sequência de interações do usuário
 durante uma mesma sessão de itens, tal que\cite{survey_wang_2021}:

 \begin{equation*}
  \begin{aligned}
  U & = \{u_1, u_2, \ldots, u_{|U|}\}, \\
  V & = \{v_1, v_2, \ldots, v_{|V|}\}, \\
  u_n & = v_n, \\
  U & = V.
\end{aligned}
\end{equation*}

$U$ é o conjunto de todos os usuários existentes. O usuário $u$ equivale ao
usuário que executa uma ação, como a criação de uma sessão. $V$ é o conjunto de
todos os $v$ itens existentes. No caso do presente trabalho, em que deseja-se
recomendar usuários a serem convidados, os conjuntos $U$ e $V$ são iguais nessa
aplicação específica.

\begin{equation*}
  \begin{aligned}
  A & = \{a_1, a_2, \ldots, a_{|A|}\}, \\
  o & = <v, a>, \\
  O & = \{o_1, o_2, \ldots, o_{|O|}\}
  \end{aligned}
  \end{equation*}
$A$ é o conjunto das ações, tais como: criação de sessão, convite endereçado a
  determinado usuário, gravação, etc. Cada $o$ interação é a tupla entre item e
  ação, enquanto que $O$ é o conjunto de todas as interações.

  A principal distinção entre SBRS e sistemas de recomendação \textit{session-aware}
  é a ausência de identificação do usuário que realiza as ações sobre os itens.
  Ou seja, no caso \textit{session-aware}:

  \begin{equation}
    o = <u, v, a>
    \end{equation}
  em que a interação é uma tupla entre usuário, item e ação.

  \begin{equation*}
  \begin{aligned}
  s & = [o_1, o_2, \ldots, o_{|s|}], \\
  S & = \{s_1, s_2, \ldots, s_{|S|}\}, \\
  \hat{l} & = [\hat{o}_1, \hat{o}_2, \ldots, \hat{o}_{|\hat{l}|}].
  \end{aligned}
  \end{equation*}
  
    Por sua vez, a sessão $s$ é uma lista de
  $|s|$ interações, tal que o conjunto de todas as sessões é dado por $S$.
  A lista de $\hat{o}$ interações recomendadas é dada por $\hat{l}$,

  O objetivo de um SBRS é selecionar uma lista recomendada de interações
  que maximize a função de desempenho $f(s, c)$ condicionada ao contexto $c$ da
  sessão $s$:
  \[\hat{l} = \arg \max f(s, c), \quad c \in C, \quad s \in S.\]   

 O contexto $c$ é definido como um conjunto de informações adicionais associadas
  à sessão e que podem ser utilizadas para melhorar a assertividade das
  recomendações.  Hora do dia, clima e localização do usuário são parâmetros os
  quais podem ser incluídos por contexto. Dois exemplos de aplicações de SBRS
  são a recomendação de músicas em uma sessão de execução \cite{music_2013} e
  produtos a serem adicionados em um carrinho de compras virtual
  \cite{shopping_cart_2023}.

\subsection{Bases de comparação não-personalizadas}

\subsection{Bases de comparação por extração de padrões}