\section{Propriedades dos sistemas de recomendação}
\subsection{\textit{Feedback} explícito e implícito}

Em geral, recomendações personalizadas são baseadas no \textit{feedback} dos
usuários, ou seja, em informações providas pelos usuários, utilizadas para
identificar suas preferências. Essas informações são ações explícitas ou dados
inferidos de forma implícita.

Historicamente, a pesquisa em de sistemas de recomendação foi dominada por
cenários nos quais os usuários declaram explicitamente seu interesse em itens,
principalmente na forma de avaliações de itens em uma escala numérica de 1 a 5.
Esses algoritmos modelam o problema como uma tarefa de preenchimento da matriz
de avaliações.

A facilidade de obter \textit{feedbacks} explícitos preenchidos por usuários
depende do domínio da aplicação. Por exemplo, em sistemas de consumo de
produtos, como comércio eletrônico ou \textit{streamings} de mídia. Por outro
lado, uma aplicação sem qualquer forma de avaliação explícita torna o sistema de
recomendação dependente de informações comportamentais monitoradas dos usuários,
ou seja, de \textit{feedback} implícito \cite{ludewig2020advances}.

O \textit{feedback} implícito do usuário é caracterizado por ações diretamente
observáveis do usuário ou sinais de preferência indiretos, relacionados a
características ou ações do
usuário.

A dificuldade na diferenciação de um sinal como interesse ou desinteresse é uma
das desvantagens do \textit{feedback} implícito. Por exemplo, quando um usuário
assiste a um vídeo curto em uma interface de \textit{Feed} como o TikTok ou
Instagram, o usuário assistir o vídeo por completo não é necessariamente
modelado como um \textit{feedback} positivo. Por sua vez, o usuário descartar
imediatamente determinado vídeo pode ser interpretado como \textit{feedback}
negativo. Apesar disso, sistemas de \textit{feedback} implícito em geral não contam
com sinais negativos \cite{ludewig2020advances,JLZ18}.

\subsection{Sistemas \textit{context-aware} e \textit{time-aware}} Abordagens
tradicionais de sistemas de recomendação ignoram a ordem e o contexto do
\textit{feedback} do usuário. Por exemplo, uma mudança nas preferências de longo
e curto prazo do usuário, a sazonalidade de um produto, um interesse pontual ou
o reinteresse em um item já consumido. A detecção e distinção dessas
preferências influenciam positivamente na assertividade das recomendações.

Padrões semelhantes são encontrados em diferentes aplicações. Nomes específicos
foram cunhados para distingui-los, como a recomendação \textit{context-aware},
\textit{time-aware} e \textit{sequence-aware}. Essas áreas compartilham
características e similaridades. Por exemplo, sistemas \textit{session-based}
são considerados um caso particular de sistemas \textit{context-aware}, em que o
contexto é simplificado e delimitado por uma sessão \cite{survey_wang_2021}.
Sistemas \textit{time-aware} são considerados um caso particular de sistemas
\textit{sequence-aware}, em que o instante de tempo de cada interação é
registrado e capaz de modelar a sequência de interações
\cite{ludewig2020advances}.

\subsubsection{\textit{Context-aware}} 

Em sistemas de recomendação tradicionais, a estimativa da utilidade de um item a
um determinado usuário é obtida em função desses dois dados. Dessa forma,
usuário e item são as duas entidades fundamentais nesses sistemas. Apesar disso,
uma série de outros fatores podem influenciar nas preferências do usuário. Por
exemplo, o dia no ano, a localização e o clima podem impactar as preferências de
compra de itens típicos associados a praias ou a invernos rigorosos. Essas
informações são consideradas fatores contextuais \cite{rec_sys_handbook_2022_context}.
Sistemas \textit{context-aware} estimam a utilidade de um item em função de três
entidades: usuário, item e contexto.

A disponibilidade dos fatores contextuais é classificada como explícita ou
latente. Contextos explícitos são informações diretamente observáveis e
mensuráveis. Contextos latentes são aqueles que não podem ser medidos
diretamente, mas podem ser inferidos a partir de modelos de fatores latentes ou
de \textit{embeddings} em modelos de aprendizado profundo.

Por sua vez, a mutabilidade dos fatores contextuais ao longo do tempo é
classificada como estática ou dinâmica. Contextos estáticos são informações com
associação e valor inalterados ao longo do tempo. Por exemplo, informações
contextuais de instante e preço de compra em um comércio eletrônico permanecem
inalterados. Contextos dinâmicos são informações que podem ser alteradas ao
longo do tempo. Por exemplo, informações de contexto do usuário vigente, no
instante em que uma recomendação é realizada
\cite{rec_sys_handbook_2022_context}.

\subsubsection{\textit{Time-aware}}

A informação da data da interação destaca-se
pela facilidade de obtê-la explicitamente, uma vez que a grande maioria dos
sistemas registram o instante de tempo associado às avaliações ou ao consumo de
determinado item. Sistemas de recomendação \textit{time-aware} são um caso
particular de sistemas \textit{context-aware}, em que o contexto é delimitado
pela dimensão temporal.

Geralmente, sistemas \textit{time-aware} utilizam o instante de tempo da
interação, porém outras fontes de informação temporal, como a data de
incorporação de um item no catálogo do sistema ou a data de registro do usuário,
tornam o sistema em questão \textit{time-aware}.

Por mais que sistemas \textit{time-aware} sejam capazes de inferir preferências
de curto e longo prazo, a dependência de informações temporais torna-os
suscetíveis a problemas de \textit{cold-start} \cite{rec_sys_handbook_2022_multi}.

\subsubsection{\textit{Sequence-aware}}
A recomendação \textit{sequence-aware} é dedicada a sequências e padrões
repetitivos no \textit{feedback} dos usuários. Sistemas \textit{session-aware} e
\textit{session-based} são classificados como uma subárea da recomendação
\textit{sequence-aware}.

Em geral, os dados de entrada em um problema \textit{sequence-aware} são listas
de interações do usuário em ordem cronológica. Nos casos em que consta o instante
de tempo de cada interação, também trata-se de um problema \textit{time-aware}.

Devido à falta de conjuntos de
dados publicamente disponíveis com informações adicionais de metadados e, ao
mesmo tempo, visando a comparabilidade e reprodutibilidade, as técnicas
apresentadas na literatura dependem apenas de um tipo de
interação e não consideram metadados adicionais.

Sistemas \textit{sequence-aware} são modelados a partir de sequências ordenadas
de interações, sem necessariamente delimitar as ações em sessões. Não há
modelagem explicita agrupando conjuntos de interações. As preferências de longo
ou curto prazo são inferidas a partir da ordem, da co-ocorrência ou da distância
entre interações \cite{sessionbaseddp, quadrana2018sequence}.


\section{Sistemas de Recomendação \textit{Session-Based} e \textit{Session-Aware}}

\subsection{Definição}

Os SBRS são aplicáveis quando os dados
 de entrada do modelo consistem em um conjunto ou uma sequência de interações do usuário
 durante uma mesma sessão de itens, tal que \cite{survey_wang_2021}:

 \begin{equation*}
  \begin{aligned}
  U & = \{u_1, u_2, \ldots, u_{|U|}\}, \\
  V & = \{v_1, v_2, \ldots, v_{|V|}\}, \\
  u_n & = v_n, \\
  U & = V.
\end{aligned}
\end{equation*}

$U$ é o conjunto de todos os usuários existentes. O usuário $u$ equivale ao
usuário que executa uma ação, como a criação de uma sessão. $V$ é o conjunto de
todos os $v$ itens existentes. No caso do presente trabalho, em que deseja-se
recomendar usuários a serem convidados, os conjuntos $U$ e $V$ são iguais nessa
aplicação específica.

\begin{equation*}
  \begin{aligned}
  A & = \{a_1, a_2, \ldots, a_{|A|}\}, \\
  o & = <v, a>, \\
  O & = \{o_1, o_2, \ldots, o_{|O|}\}
  \end{aligned}
  \end{equation*}
$A$ é o conjunto das ações, tais como: criação de sessão, convite endereçado a
  determinado usuário, gravação, etc. Cada $o$ interação é a tupla entre item e
  ação, enquanto que $O$ é o conjunto de todas as interações.

  A principal distinção entre SBRS e sistemas de recomendação \textit{session-aware}
  é a ausência de identificação do usuário que realiza as ações sobre os itens.
  Ou seja, no caso \textit{session-aware}:

  \begin{equation}
    o = <u, v, a>
    \end{equation}
  em que a interação é uma tupla entre usuário, item e ação.

  \begin{equation*}
  \begin{aligned}
  s & = [o_1, o_2, \ldots, o_{|s|}], \\
  S & = \{s_1, s_2, \ldots, s_{|S|}\}, \\
  \hat{l} & = [\hat{o}_1, \hat{o}_2, \ldots, \hat{o}_{|\hat{l}|}].
  \end{aligned}
  \end{equation*}
  
    Por sua vez, a sessão $s$ é uma lista de $|s|$ interações, tal que o
  conjunto de todas as sessões é dado por $S$. A lista de $\hat{o}$ interações
  recomendadas é dada por $\hat{l}$.

  O objetivo de um SBRS é selecionar uma lista recomendada de interações
  que maximize a função de desempenho $f(s, c)$ condicionada ao contexto $c$ da
  sessão $s$:
  \[\hat{l} = \arg \max f(s, c), \quad c \in C, \quad s \in S.\]   

 O contexto $c$ é definido como um conjunto de informações adicionais associadas
  à sessão e que podem ser utilizadas para melhorar a assertividade das
  recomendações.  Hora do dia, clima e localização do usuário são parâmetros os
  quais podem ser incluídos por contexto.

  É importante destacar que $s$ e $S$ podem
  ser definidos com um conjunto de interações desprovido de ordenação temporal
  ou como listas ordenadas de interações e de sessões. Essas duas abordagens
  dependem das propriedades \textit{context-aware}, \textit{time-aware}
  e \textit{sequence-aware} do sistema de recomendação.

  No caso das recomendações \textit{session-based} e \textit{session-aware},
  cada interação é associada a um identificador de sessão, agrupando várias
  interações consecutivas. As interações são agrupadas em uma única sessão até
  que o usuário não execute nenhuma interação por um período específico de
  tempo, geralmente após trinta minutos de ociosidade.
  
  Ao contrário da recomendação \textit{sequence-aware}, os recomendadores
  \textit{session-aware} necessariamente consideram a sessão atual do usuário.
  Na recomendação \textit{sequence-aware}, algumas técnicas dependem apenas do
  histórico de longo prazo dos usuários, levando em consideração a ordem dos
  eventos.

  % \subsection{Características}


% A recomendação \textit{session-based} pode ser interpretada tanto como
% \textit{sequence-aware} quanto \textit{session-aware} em cenários de
% \textit{cold-start} do usuário, ou seja, na ausência de informações históricas
% para todos os usuários. Por outro lado, a recomendação \textit{session-based} é
% desprovida de modelos personalizados de longo prazo. A personalização das
% sugestões de itens depende exclusivamente das últimas interações do usuário.

  \subsection{Avaliação de desempenho em SBRS}
  A ideia geral da avaliação em SBRS consiste em revelar uma parte das interações
  iniciais de uma sessão e comparar as interações recomendadas subsequentes com
  as interações que o usuário realmente realizou.
  
  Duas abordagens são utilizadas na literatura para avaliar o desempenho de
  sistemas \textit{session-based} e \textit{session-aware}: \textit{next-item}
  e \textit{remaining items}.

  Na abordagem \textit{next-item}, as primeiras $N$ interações de uma sessão de
  teste são reveladas ao sistema de recomendação e apenas a interação
  subsequente é utilizada como prova real e comparada com o valor recomendado.
  Dessa forma, o sistema pode revelar item a item e avaliar cada recomendação,
  ou revelar apenas o último item de cada sessão.

  Por sua vez, a abordagem \textit{remaining items} revela todos os itens
  subsequentes ao realizar a prova real. O valor recomendado é comparado
  com todos os itens subsequentes, e não apenas com o próximo item.

  \abbrev{HR}{\textit{Hit Rate}} \abbrev{MAP}{\textit{Mean Average Precision}}
  \abbrev{MAR}{\textit{Mean Average Rank}} \abbrev{NDCG}{\textit{Normalized
  Discounted Cumulative Gain}} Para a abordagem \textit{next-item}, as métricas
  de avaliação utilizadas são \textit{Hit Rate} (HR) e \textit{Mean Reciprocal
  Rank} (MRR). Para a abordagem \textit{remaining items}, as métricas de
  avaliação utilizadas são a Precisão, \textit{Recall}, \textit{Mean Average
  Precision} (MAP), \textit{Mean Average Rank} (MAR), \textit{Normalized
  Discounted Cumulative Gain} (NDCG) e \textit{F1} \cite{sessionbaseddp}.

  Uma vez que sistemas de recomendação tendem a enviesar as recomendações para
  itens populares, as métricas de cobertura e popularidade são utilizadas para
  avaliar a diversidade e o viés das recomendações. Sessões assertivas com pouco
  viés de popularidade e maior cobertura indicam maior qualidade na
  personalização das recomendações.

  \subsubsection{\textit{Hit Rate} (HR)}
  O HR@N é a porcentagem de vezes em que um item relevante
  é recomendado entre os $N$ primeiros itens da lista de recomendações:

  \begin{equation}
    % \text{HR}_{\text{N}} = \frac{1}{|U|} \sum_{u \in U} \sum_{i \in R_{u, N}} \frac{1}{N}
    \text{HR@N} = \frac{1}{Q}\sum_{i=1}^{Q}\begin{cases}
      1 & \text{se} \quad i \leq N \\
      0 & \text{caso contrário}
    \end{cases}
  \end{equation}
  em que $Q$ é o comprimento da lista de itens recomendados para uma determinada
  predição e $i$ é a posição do item na lista de recomendações.
  \subsubsection{\textit{Mean Reciprocal Rank} (MRR)} O MRR@N é similar ao HR@N,
  porém leva em consideração a posição do item. Se o item recomendado
  corretamente consta nas primeiras posições da lista de recomendações, o MRR@N
  é maior, decaindo conforme a posição do item recomendado corretamente fica
  mais distante do início da lista de recomendações. Por sua vez, o incremento
  no HR@N em razão de um item recomendado no início ou no final da lista contém
  pesos idênticos.

  \begin{equation}
    \text{MRR@N} = \frac{1}{Q}\sum_{i=1}^{Q}\begin{cases}
      \dfrac{1}{i} & \text{se} \quad i \leq N \\
      0 & \text{caso contrário}
    \end{cases}
  \end{equation}

  \subsubsection{Precisão e \textit{Recall}}
  No contexto de classificadores, a precisão e o \textit{recall} avaliam
  a qualidade de um modelo a partir da matriz de confusão descrita
  na tabela \ref{tab:confusion_matrix}::

  \begin{align}
    \text{Precisão} &= \frac{\text{VP}}{\text{VP} + \text{FP}} \\
    \text{\textit{Recall}} &= \frac{\text{VP}}{\text{VP} + \text{FN}}
  \end{align}

  \begin{table}[H]
    \centering
    \begin{tabular}{c|c|c|}
      \cline{2-3}
      & \multicolumn{2}{c|}{\textbf{Predição}} \\ \cline{2-3} 
      & \textbf{Positivo} & \textbf{Negativo} \\ \hline
      \multicolumn{1}{|c|}{\textbf{Positivo}} & Verdadeiro Positivo (VP) & Falso Positivo (FP) \\ \hline
      \multicolumn{1}{|c|}{\textbf{Negativo}} & Falso Negativo (FN) & Verdadeiro Negativo (VN) \\ \hline
    \end{tabular}
    \caption{Matriz de confusão}
    \label{tab:confusion_matrix}
  \end{table}
  \abbrev{VP}{Verdadeiro Positivo}
  \abbrev{FP}{Falso Positivo}
  \abbrev{FN}{Falso Negativo}
  \abbrev{VN}{Verdadeiro Negativo}

  No contexto de sistemas de recomendação, a Precisão@N avalia a quantidade de
  itens relevantes em relação ao total de $N$ itens recomendados na lista. O conjunto
  dos $N$ itens recomendados equivale aos verdadeiros e falsos positivos:

  \begin{equation}
    \text{Precisão@N} = \frac{\text{Qntd. de itens relevantes nas primeiras N recomendações}}{N}
  \end{equation}

  Por sua vez, o Recall@N equivale a quantidade de itens relevantes entre
  os $N$ primeiros itens da lista de recomendações, desconsiderando a
  posição do item na lista. O conjunto dos itens relevantes equivale
  aos verdadeiros positivos e falsos negativos:

  \begin{equation}
    \text{Recall@N} = \frac{\text{Qntd. de itens relevantes nas primeiras N recomendações}}{\text{Quantidade de itens relevantes}}
  \end{equation}

  \subsubsection{\textit{Mean Average Precision} (MAP)} O MAP@N calcula a média
  da precisão média (AP@N) para cada consulta. Após cada item relevante ser
  recomendado, a precisão é medida, tal que a média é calculada por:

  \begin{align}
    \text{AP@N} &= \frac{\sum_{i = 1}^{K}\text{Precisão@k} \times \text{rel}_{k}}{\text{qntd. de resultados relevantes}} \\
    \text{rel}_k &= \begin{cases}
      1 & \text{se} \quad k \leq N \\
      0 & \text{caso contrário} 
    \end{cases} \\
    \text{MAP@N} &= \frac{1}{Q}\sum_{q = 1}^{Q} \text{AP@N}_{q}
  \end{align}
  em que $K$ é a quantidade de itens recomendados, $\text{rel}_{k}$ é a
  relevância do item na posição $k$ e $Q$ é a quantidade de consultas.

  \subsubsection{\textit{Normalized Discounted Cumulative Gain} (NDCG)}
  O NDCG é baseado no ganho cumulativo (CG). O ganho cumulativo é a soma dos
  valores de relevância graduais de todos os itens em uma lista de
  recomendações:

  \begin{equation}
    \text{CG@N} = \sum_{i = 1}^{N} \text{rel}_{i}
  \end{equation}
  
  \abbrev{CG}{\textit{Cumulative Gain}}
  \abbrev{DCG}{\textit{Discounted Cumulative Gain}}
  \abbrev{IDCG}{\textit{Ideal Discounted Cumulative Gain}}
  Por sua vez, o NDCG é a razão entre o ganho cumulativo
  descontado (DCG) e o ganho cumulativo descontado ideal (IDCG). Ambos contém
  um fator de penalidade no denominador:

  \begin{align}
    \text{DCG@N} &= \sum_{i = 1}^{N} \frac{2^{\text{rel}_{i}} - 1}{\log_{2}(i + 1)} \\
    \text{IDCG@N} &= \sum_{i = 1}^{\text{REL}_N} \frac{\text{rel}_{i}}{{\log_{2}(i + 1)}} \\
    \text{NDCG@N} &= \frac{\text{DCG@N}}{\text{IDCG@N}}
  \end{align}
  em que $\text{REL}_N$ é a lista de itens relevantes contidos em $N$, ordenados
  por relevância.


  \subsubsection{F1}
É calculado a partir da média harmônica entre a precisão e o \textit{recall}. Dessa forma,
um valor baixo ou próximo de zero para qualquer uma dessas duas métricas derruba
o valor do F1:

\begin{equation}
  \text{F1} = \frac{2 \times \text{Precisão} \times \text{Recall}}{\text{Precisão} + \text{Recall}}
\end{equation}

  \subsubsection{Cobertura}
  A cobertura mede a proporção de itens recomendados em relação ao total de
  itens disponíveis. Se definirmos o conjunto de itens disponíveis como $I$ e o
  conjunto de itens recomendados como $I_{p}$, a cobertura é dada por:
  \begin{equation}
    \text{Cobertura@N} = \frac{|I_{p}|}{|I|}.
  \end{equation}

  % \subsection{Exemplos}

  % Dois exemplos de aplicações de SBRS
  % são a recomendação de músicas em uma sessão de execução \cite{music_2013} e
  % produtos a serem adicionados em um carrinho de compras virtual
  % \cite{shopping_cart_2023}.

% \subsection{Bases de comparação não-personalizadas}

% \subsection{Bases de comparação por extração de padrões}