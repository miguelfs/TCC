\section{Espectro de Modulação}

O espectro de modulação, por vezes chamado de espectrograma de modulação, é o
nome dado a representações que evidenciam frequências associadas a modulações
presentes no sinal. \citet{greenberg1997} apresentam o espectrograma de
modulação em 1997 como ``\textit{um novo formato de representação para a fala}
[...] \textit{que exibe e codifica o sinal em termos da distribuição de
modulações lentas em função do tempo e da frequência}''. Trabalhos anteriores
fizeram uso do espectro da envoltória do sinal e sua correspondência com a
inteligibilidade de fala, como \citet{houtgast1973} \citet{houtgast1980}
\cite{houtgast1985} a partir de 1973 e \citet{drullman1994} em 1994.

O método apresentado por \citet{greenberg1997} consiste em obter a envoltória
nas parcelas filtradas de um sinal por sub-bandas de um banco de filtros, para
então calcular suas transformadas de Fourier. Essa abordagem também é utilizada
para o espectro de modulação apresentado por \citet{atlas2004} \cite{atlas2003}
a partir de 2003, definido como ``\textit{uma representação bidimensional} [...]
\textit{onde a primeira dimensão é a \todo{well-known} célebre frequência acústica e a segunda
dimensão é a frequência de modulação}''. A principal novidade, nesse segundo
caso, é a estimação e decomposição das partes moduladora e portadora de forma
coerente, em vez da decomposição por módulo e fase.

Em uma abordagem distinta da tradicional,
\citet{paliwal2010}\cite{so2011modulation}\cite{paliwal2012mmse} utiliza a STFT 
em vez do banco de filtros, de forma que as trajetórias dos módulos do espectro
acústico no tempo são adequadas para o cálculo do espectro de modulação de tempo
curto.

Para compreender a representação do espectrograma de modulação, definem-se dois
domínios, na bibliografia~\cite{paliwal2015}. O domínio acústico é uma
representação bidimensional de um sinal na qual, para as frequências $f$ e os
instantes de tempo $t$, mapeiam-se os respectivos coeficientes complexos de
Fourier~\cite{muller2015}:\todo{chamo de STFT?}
\symbl{$X(t,f)$}{sinal no domínio acústico}
\symbl{$\mathbb{R}$}{conjunto dos números reais}
\symbl{$\mathbb{C}$}{conjunto dos números complexos}
\begin{equation}
X(t, f): \mathbb{R}  \times  \mathbb{R} \rightarrow \mathbb{C}.
\end{equation}
O chamado espectrograma de \todo{modulo?} amplitude é a representação do módulo
desse domínio:
\begin{equation}
\mathrm{Spec}(t, f) = \left\lvert X(t, f)\right\rvert.
\end{equation}\symbl{$X(t, f)$}{sinal no domínio acústico}

Por sua vez, o domínio da modulação é capaz de representar esse mesmo sinal a
partir das mesmas coordenadas do domínio acústico, acrescidas da frequência
$\eta$ de oscilação de sua envoltória, mapeando-o também em coeficientes
complexos de Fourier:
\begin{equation}
\mathcal{X}(\eta, f, t): \mathbb{R}  \times \mathbb{R}  \times  \mathbb{R} \rightarrow \mathbb{C}.
\end{equation}\symbl{$\mathcal{X}(\eta, f, t)$}{sinal no domínio da modulação}

O correspondente espectrograma de modulação é dado por
\begin{equation}
      \mathrm{Mod Spec}(\eta, f, t) = \left\lvert \mathcal{X}(\eta, f, t)\right\rvert.
\end{equation}

A seguir, serão apresentadas duas possibilidades de implementação: uma com banco
de filtros, e outra com a STFT.

Na implementação com banco de filtros, o espectro de modulação $P_k[\mu]$ segue da
seguinte forma \cite{toolbox2010}:

\begin{equation} \label{modspec_filterbank}
    P_k[\mu] = \sum_{n = 0}^{N - 1} w[n]m_k[n] e^{-j \frac{2 \pi \mu}{N} n}
\end{equation}
em que $k$ é a k-ésima sub-banda do banco de filtros, $\mu$ é a frequência de
modulação, $w[n]$ é a função-janela com tamanho de $N$ amostras e $m_k[n]$ é a
parcela moduladora obtida pela demodulação da k-ésima sub-banda. A estimação das
partes portadora e moduladora também é utilizada na filtragem de modulação,
descrita na secção \ref{section_fil_mod}.

Ao se utilizar a STFT, calcula-se $X[l,k]$ a partir do sinal $x[n]$, da taxa de
amostragem do sinal, da função janela $w[n]$ de análise, do comprimento $M$ do
\textit{frame} em amostras, do número $N$ de amostras da DFT, e do salto da janela
em $H$ amostras~\cite{paliwal2015,mauricio}:
 \begin{equation}
      X[l, k] = \sum_{n = 0}^{M-1} x[n + l H] w[n]{\mathrm{e}^{-2\mathrm{j}\pi n k/N}}. 
 \end{equation}
 
Dessa forma, obtém-se os coeficientes de Fourier para cada \textit{bin} $k$ na
 frequência e \textit{frame} $l$ no tempo, análogos às dimensões de
 $f(t,\omega)$.

Finalmente, obtêm-se os coeficientes complexos no domínio da modulação
aplicando-se uma segunda STFT sobre o módulo das séries de \textit{frames}, para
cada \textit{bin} $k$ do domínio acústico~\cite{paliwal2015}:
\begin{equation}
      % \label{eq2}
      % \eqref{eq2}
      \mathcal{X}[\ell, k, \mu] = \sum_{l = 0}^{\mathcal{M}-1}\left\lvert X[l+\ell \mathcal{H},k] \right\rvert v[l]{\mathrm{e}^{-2\mathrm{j}\pi \mu l/\mathcal{N}}},      
\end{equation}
no qual $\ell$ é o índice do \textit{frame} no domínio da modulação, $k$ é o
 índice da frequência advinda do domínio acústico, $\mu$ é o índice do
 \textit{bin} da frequência de modulação, $\mathcal{N}$ é o número de amostras
 da DFT, $\mathcal{M}$ é o tamanho do \textit{frame} em amostras e $\mathcal{H}$
 é o deslocamento da janela de análise $v[l]$ no domínio da modulação, em
 amostras. O espectrograma de modulação é dado por
\begin{equation}
\mathrm{ModSpec}[\ell, k, \mu] = \left\lvert \mathcal{X}[\ell, k, \mu]\right\rvert. 
\end{equation}

\todo{inversa?}
\todo{imagens?}
\todo{specs de implementação}

