\section{Introdução}

A primeira publicação sobre sistemas de recomendação é creditada a
\citet{rich1979user}. A autora apresenta, em 1979, um sistema para recomendação de livros
em uma biblioteca. O sistema, chamado \textit{Grundy}, realiza uma série de
perguntas abertas ao usuário sobre suas características e personalidade, com o objetivo
de identificar as afinidades entre ele e o conteúdo recomendado. Caso as
respostas contenham palavras específicas de gatilho para o sistema, é atribuído
um estereótipo ao modelo de usuário que representa o interlocutor. A atribuição
de uma série de estereótipos atualiza os valores nas dimensões do modelo de
usuário, que são utilizadas para recomendar livros cujos modelos tenham valores
similares.

\vspace{2em}

\begin{table}[h!]
\begin{center}
   \begin{tabular}{|m{3cm}|m{2cm}|m{2cm}|}
       \hline
       \textbf{FACET} & \textbf{Value} & \textbf{Rating} \\
       \hline
       Activated by & ``\textit{athletic}'' & \\
       Gender & any person & \\
       \hline
       Motivations: & & \\
       \hspace{1mm} Excite & 3 & 600 \\
       \hline
       Interests: & & \\
       \hspace{1mm} Sports & 4 & 800 \\
       \hline
   \end{tabular}
   \caption{Exemplo de esterótipo do sistema de recomendação Grundy.}
\end{center}
\end{table}

A abordagem de filtragem colaborativa, popularizada a partir dos anos 90 e
incluída pela Amazon \cite{amazon2017} em seu comércio eletrônico, possui
algumas diferenças em relação ao sistema proposto por Rice, segundo a própria
autora \cite{rich:homepage}:

\begin{quotation}\small\noindent \textit{``Na filtragem colaborativa, a única
   coisa que o sistema sabe sobre os livros é quem os comprou. A única coisa que
   ele sabe sobre um indivíduo é quais livros ele comprou. Ele não sabe o motivo
   pelo qual a pessoa gosta daqueles livros em particular. Porém, ele sabe de
   milhares de outras pessoas que gostaram dos mesmos livros e sabe quais outros
   livros essas pessoas gostaram. Assim, ele pode recomendar novos livros a um
   usuário sem nenhum modelo aprofundado das preferências dele ou dos livros.
   Sistemas de filtragem colaborativa substituem o banco de dados menor e mais
   aprofundado de Grundy com um banco de dados superficial e massivo."}
   \end{quotation}


% A popularização dos sistemas de recomendação ocorre a partir dos anos 90,
% conjuntamente com a democratização no uso da internet e o aparecimento das
% primeiras grandes empresas que se tornaram referência no uso de sistemas de
% recomendação e que dependem deles para maior sucesso de seus negócios.

% A popularização dos sistemas de recomendação ocorre a
% partir dos anos 90, conjuntamente com a popularização da internet.

% Os objetivos de um sistema de recomendação são (aggrawal):
% \begin{itemize}
%    \item Relevância: garantir que as recomendações sejam adequadas e úteis para
%    o usuário, levando em consideração suas preferências e necessidades. Não é
%    suficiente que as recomendações tenham bom desempenho em avaliação
%    \textit{offline}, é necessário que realmente faça sentido para o usuário
%    final.
%    \item Acurácia: ?
%    \item Cobertura: garantir que o sistema não fique restrito a recomendar uma
%    pequena parcela da base disponível.
%    \item Novidade: apresentar ao usuário itens ou informações que sejam novos,
%    que ainda não foram consumidos ou que sejam desconhecidos dele.
%    \item Serendipidade: surpreender o usuário com recomendações inesperadas e
%    relevantes, que vão além das preferências e padrões usuais e óbvios,
%    explorando associações que não sejam explicitas nos dados de recomendação.
%    \item Diversidade: oferecer ao usuário maior variedade de opções, evitando
%    que recomendações muito similares possam ser desaprovadas pelo usuário caso
%    alguma delas não o agrade.
%    \item Confiança e Trust:?
%    \item Robustez e Estabilidade:
%    \item escalabilidade.
% \end{itemize}

% Para a maioria dos objetivos acima, há uma métrica de avaliação correspondente,
% que será descrita a seguir.