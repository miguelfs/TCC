\section{Modulação em Amplitude}
A modulação em amplitude (AM\abbrev{AM}{modulação em amplitude}, do inglês
\textit{Amplitude Modulation}) é o processo no qual a envoltória de uma onda
portadora oscila em torno de um valor médio, conforme um sinal modulador em
baixa frequência~\cite{haykin2008}. No contexto de telecomunicações, a
moduladora é um sinal em banda base, isto é, contém a informação em um espectro
delimitado a ser transmitida.

Considerando a portadora como um tom puro de frequência $f_c$ e amplitude $A_c$:
\symbl{$c(t)$}{portadora no domínio contínuo}
\symbl{$f_c$}{frequência da portadora}
\symbl{$A_c$}{amplitude da portadora}
\begin{equation}
    c(t) = A_c \cos(2\pi f_c t).
\end{equation}

    Um sinal $s(t)$ modulado em amplitude é definido pela composição de suas
parcelas portadora $c(t)$ e moduladora $m(t)$:
\symbl{$k_a$}{sensibilidade à amplitude}
\symbl{$m(t)$}{moduladora no domínio contínuo}
\symbl{$s(t)$}{sinal modulado no domínio contínuo}
\begin{equation}
    s(t) = [1+k_a m(t)]c(t) = A_c[1+k_a m(t)]\cos(2\pi f_c t) \label{eq:mod_1}
\end{equation}
em que $k_a$ é a sensibilidade à amplitude da moduladora. Para que a envoltória
dada pela função $1+k_a m(t)$ seja sempre positiva, evitando reversão de fase em
decorrência de sobremodulação, basta que
\begin{equation}
    |k_a m(t)| < 1, \quad \forall t.
\end{equation}

No domínio da frequência, a equação \eqref{eq:mod_1} é dada por:
\symbl{$M(f)$}{moduladora no domínio da frequência}
\symbl{$S(f)$}{sinal modulado no domínio da frequência}
\symbl{$\delta (t)$}{impulso unitário}
\begin{equation}
    S(f) = \frac{A_c}{2} [\delta (f - f_c) + \delta(f + f_c)] + \frac{k_a A_c}{2} [M(f - f_c) + M (f + f_c)], \label{eq:mod_2}
\end{equation}

em que $\delta$ é o impulso unitário e $M(f)$ é a parcela moduladora no domínio
da frequência.

Pela equação \eqref{eq:mod_2}, observa-se que a parcela com impulsos unitários
não contribui diretamente para a transmissão da informação contida em $M(f)$,
além de despender grande parte da potência do sinal. Uma segunda característica
de $S(f)$ \todo{add plot}é a existência de dois lóbulos simétricos no domínio da
frequência, o que confere redundância na transmissão do sinal e maior largura de
banda. Para solucionar essas duas características indesejáveis em
telecomunicações, foram criadas as modulações com banda lateral dupla e
supressão de portadora (DSB-SC, do inglês \textit{double-sideband
suppressed-carrier}), banda lateral vestigial (VSB, do inglês \textit{vestigial
sideband modulation}) e banda lateral suprimida (SSB, do inglês
\textit{single-sideband modulation}), que vão além do escopo desse trabalho.

\abbrev{DSB-SC}{banda lateral dupla com supressão de portadora}
\abbrev{VSB}{banda lateral vestigial}
\abbrev{SSB}{banda lateral suprimida}

Uma notação mais simples para a equação \eqref{eq:mod_1}, presente na maior
parte da bibliografia, em que $k_a = 1$ e que desconsidera o \textit{offset}, é
dada pelo produto dos termos:
\begin{equation}
    s(t) = m(t)c(t). \label{eq:signal_eq}
\end{equation}
No domínio do tempo discreto,
\symbl{$m \lbrack n \rbrack$}{moduladora no domínio discreto}
\symbl{$c \lbrack n \rbrack$}{portadora no domínio discreto}
\symbl{$s \lbrack n \rbrack$}{sinal modulado no domínio discreto}
\begin{equation}
    s[n] = m[n]c[n]. 
\end{equation}

As parcelas moduladoras $m(t)$ e $m[n]$ podem assumir valores complexos
 \cite{schimmel2007}, apesar do termo \textit{envoltória} referir-se,
 geralmente, à parcela de valores reais não-negativos \cite{haykin2008} com
 lenta variação sobre um sinal. No caso particular em que $m(t)$ é um sinal de
 valores reais não-negativos, para a equação \eqref{eq:signal_eq}, temos que:

\begin{equation}
    m(t) \geq s(t), \forall t.
\end{equation}
A decomposição do sinal em envoltória e portadora em valores complexos é
abordada por \citet{atlas2004}.\todo{explorar mais}