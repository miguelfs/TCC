\section{Sistemas de Recomendação}
A primeira publicação sobre sistemas de recomendação é creditada a
\citet{rich1979user}. A autora apresenta, em 1979, um sistema para recomendação
de livros em uma biblioteca. O sistema, chamado \textit{Grundy}, realiza uma
série de perguntas abertas ao usuário sobre suas características e
personalidade, com o objetivo de identificar as afinidades entre o usuário e o
conteúdo recomendado. Caso as respostas contenham palavras específicas de
gatilho para o sistema, é atribuído um estereótipo ao modelo de usuário, que
incrementa valores em dimensões do modelo de usuário, em que cada dimensão
corresponde a uma faceta.

\vspace{2em}
\begin{table}[h!]
   \begin{center}
      \begin{tabular}{|m{3cm}|m{2cm}|m{2cm}|}
          \hline
          \textbf{FACET} & \textbf{Value} & \textbf{Rating} \\
          \hline
          Activated by & ``\textit{athletic}'' & \\
          Gender & any person & \\
          \hline
          Motivations: & & \\
          \hspace{1mm} Excite & 3 & 600 \\
          \hline
          Interests: & & \\
          \hspace{1mm} Sports & 4 & 800 \\
          \hline
      \end{tabular}
      \caption{Exemplo de esterótipo do sistema de recomendação Grundy. O estereótipo de atleta incrementa uma faceta de motivação e outra de interesse do usuário.}
      \label{grundy}
   \end{center}
   \end{table}

A atribuição de uma série de estereótipos atualiza os valores nas dimensões do
modelo de usuário, utilizadas para recomendar um livro de cada vez. A aplicação
segue as seguintes regras para gerar uma recomendação:

\begin{enumerate}
   \item Seleciona uma faceta no modelo do usuário com valores altos.
   \item Busca por livros os quais tenham essa faceta associada com valor similar.
   \item Para cada livro dessa faceta específica, compara as demais facetas
   do modelo de usuário com as dimensões do livro. Elimina livros que excedam
   limites estabelecidos pelo usuário, como tolerância para violência.
   \item Dos livros que não foram eliminados, escolhe o que melhor corresponde.
\end{enumerate}

No exemplo da tabela \ref{grundy}, um usuário que se identifique com o estereótipo
``\textit{atleta}'', independente de seu gênero, terá um incremento em suas
motivações e interesses proporcionais à sua correspondência com o estereótipo.
Dessa forma, leituras relacionadas a esportes ou que gerem excitação serão
mais recomendadas a esse usuário.

A filtragem colaborativa, popularizada a partir dos anos 90 e utilizada pela
Amazon \cite{amazon2017} para realizar recomendações,
possui algumas diferenças em relação ao sistema proposto por Rice, segundo a
própria autora \cite{rich:homepage}:

\begin{quotation}\small\noindent \textit{``Na filtragem colaborativa, a única
   coisa que o sistema sabe sobre os livros é quem os comprou. A única coisa que
   ele sabe sobre um indivíduo é quais livros ele comprou. Ele não sabe o motivo
   pelo qual a pessoa gosta daqueles livros em particular. Porém, ele sabe de
   milhares de outras pessoas que gostaram dos mesmos livros e sabe quais outros
   livros essas pessoas gostaram. Assim, ele pode recomendar novos livros a um
   usuário sem nenhum modelo aprofundado das preferências dele ou dos livros.
   Sistemas de filtragem colaborativa substituem o banco de dados menor e mais
   aprofundado de Grundy com um banco de dados superficial e massivo.''}
   \end{quotation}
