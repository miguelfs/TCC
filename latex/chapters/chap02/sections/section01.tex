\section{Modulação em Amplitude e Demodulação}
A modulação em amplitude (AM, do inglês \textit{Amplitude Modulation}) é o
processo no qual a envoltória de uma onda portadora oscila em torno de um valor
médio, conforme um sinal modulador em baixa frequência~\cite{haykin2008}. No
contexto de comunicações, a moduladora é um sinal em banda básica, isto é,
contém a informação em um espectro delimitado a ser transmitida.

Considerando a portadora como um tom puro de frequência $f_c$ e amplitude $A_c$:
\begin{equation}
    c(t) = A_c \cos(2\pi f_c t).
\end{equation}

    Um sinal $s(t)$ modulado em amplitude é definido pela composição de
suas parcelas portadora $c(t)$ e moduladora $m(t)$:
\begin{equation}
    s(t) = [1+k_a m(t)]c(t) = A_c[1+k_a m(t)]\cos(2\pi f_c t)
\end{equation}
em que $k_a$ é a sensibilidade de amplitude da moduladora. Para que a envoltória
dada pela função $1+k_a m(t)$ seja sempre positiva, evitando reversão de fase
em decorrência de sobremodulação, basta que
\begin{equation}
    |k_a m(t)| < 1, \quad \forall t.
\end{equation}

Uma notação mais simples, presente na maior parte da
bibliografia, em que $k_a = 1$ e que desconsidera o
\textit{offset}, é dada por
\begin{equation}
    s(t) = m(t)c(t).
\end{equation}
No domínio do tempo discreto,
\begin{equation}
    s[n] = m[n]c[n]. 
\end{equation}
\symbl{$c(t)$}{portadora no domínio contínuo}
\symbl{$f_c$}{frequência da portadora}
\symbl{$A_c$}{amplitude da portadora}
\symbl{$m(t)$}{moduladora no domínio contínuo}
\symbl{$s(t)$}{sinal modulado no domínio contínuo}
\symbl{$k_a$}{sensibilidade de amplitude}
\symbl{$m \lbrack n \rbrack$}{moduladora no domínio discreto}
\symbl{$c \lbrack n \rbrack$}{portadora no domínio discreto}
\symbl{$s \lbrack n \rbrack$}{sinal modulado no domínio discreto}