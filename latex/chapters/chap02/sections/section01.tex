\section{Modulação em Amplitude e Demodulação}
A modulação em amplitude (AM, do inglês Amplitude Modulation) é o processo no
qual a amplitude de uma onda portadora varia linearmente em torno de um valor
médio, conforme um sinal modulador em banda-base, isto é, conforme a banda de
frequência que a fonte original de informação ocupa.


requency band occupied by an original source of information.

Um sinal $x(t)$ modulado em amplitude é definido pela composição de suas parcelas portadora e moduladora:

\begin{equation}
    c(t) = A_c \cos(2\pi f_c t)
\end{equation}
\begin{equation}
    x(t) = c(t)m(t)
\end{equation}
% Lathi
% Haykin