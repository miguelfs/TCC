\section{Filtragem de Modulação} \label{section_fil_mod} A filtragem de
modulação é uma técnica para filtragem de envoltórias com oscilações lentas,
presentes em sub-bandas de frequência, aplicada sobre sinais
não-estacionários.\cite{clark2009sum} \cite{li2005properties}. O objetivo, em
aplicações de fala, é obter um sinal mais inteligível na saída, composto apenas
pela composição de tons com modulações de baixa frequência associadas à
articulação vocal, eliminando as demais parcelas, as quais, quando presentes,
podem conter ruído e artefatos que degradam a compreensão da fala.

Essa técnica assume que um sinal $x[n]$ de valores reais é
representado pela soma dos produtos entre moduladora e portadora que compõem
cada sub-banda $x_k[n]$:
\begin{equation}
    x[n] = \sum_{k = 0}^{K - 1} x_k[n] = \sum_{k = 0}^{K - 1} m_k[n] c_k[n]
\end{equation}

A implementação genérica da filtragem de modulação está representada no diagrama
de blocos abaixo.

\begin{figure}[h]
    \centering
    \begin{tikzpicture}[auto,>=latex']
        \tikzstyle{block} = [draw, shape=rectangle, minimum height=3em, minimum
        width=3em, node distance=4cm, line width=1pt]
        %Creating Blocks and Connection Nodes
        \node at (-2.5,0) (input) {$x[n]$};
        \node [block] (filterbank) {Banco de filtros};
        \node [block, right of=filterbank] (dem) {demodulação};
        \node [block, right of=dem] (LP) {passa-baixas};
        \node [block, below of = filterbank] (sintese) {síntese};
        \node [block, below of=dem] (res) {reconstrução};
        \node [left of=sintese, node distance=2.5cm] (output) {$x'[n]$};
        %Connecting Blocks
        \begin{scope}[line width=1pt]
            \draw[->] (input) -- (filterbank);
            \draw[->, line width=1mm] (filterbank) -- node[name=subband] {$x_k[n]$} (dem);
            \draw[->, line width=1mm] (dem) -- node[name=mod] {$m_k[n]$} (LP);
            \draw[->, line width=1mm] (dem) -- node[name=carrier] {$c_k[n]$} (res);
            \draw[->, line width=1mm] (LP) |- node[name=new_mod] {$m'_k[n]$} (res);
            \draw[->, line width=1mm] (res) -- node[name=new_subband] {$x'_k[n]$} (sintese);
            \draw[->] (sintese) -- (output);
        \end{scope}
    \end{tikzpicture}
    \caption{Filtro de modulação, em diagrama de blocos. Adaptado de \cite{li2005properties}.}
\end{figure}

O sinal banda-larga $x[n]$ passa por um banco de filtros, no qual as  $x_k[n]$
saídas do banco de filtros passam por um módulo de demodulação, no qual são
separadas as parcelas moduladoras $m_k[n]$ e portadoras $c_k[n]$ de cada
sub-banda. Cada parcela $m_k[n]$ é filtrada por um filtro passa-baixas, cuja
resposta ao impulso é $g[n]$. Em seguida, cada sub-banda é reconstruída a partir da portadora
$c_k[n]$ e da modulação filtrada $m'_k[n]$. Finalmente, o sinal de banda larga
$x'_k[n]$ é sintetizado com as sub-bandas $x'_k[n]$ reconstruídas, obtidas a
partir das modulações filtradas.

Em aplicações de \textit{speech enhancement}, o projeto do banco de filtros
emula a resposta em frequência da cóclea no sistema auditivo. Por sua vez, a
frequência de corte do filtro passa-baixas é tipicamente de 16Hz
\cite{drullman1994}, tal que frequências abaixo desse valor estão associadas às
periodicidades na construção de frases, palavras e sílabas.

Uma forma de calcular $m'_k[n]$ é a partir do espectro de modulação $P_k[\mu]$,
apresentado na equação \ref{modspec_filterbank}:

\begin{equation}
    m'_k[n] = g[n] \ast m_k[n] = \sum_{q = 0}^{n} g[n - q] m_k[q]
\end{equation}

\begin{equation}
    m'_k[n] = \sum_{\mu=0}^{N-1} G[\mu]P_k[\mu] e^{j \frac{2 \pi \mu}{N} n}
\end{equation}  \todo{janela na eq?}

Dessa forma, obtém-se $m'_k[n]$ a partir da transformada inversa do espectro de
modulação filtrado por $G[\mu]$ \cite{toolbox2010}.

\todo{specs de implementação}