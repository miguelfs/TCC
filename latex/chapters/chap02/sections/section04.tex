\section{Filtragem de Modulação}
A filtragem de modulação é uma técnica para filtragem de envoltórias com
oscilações lentas, presentes em sub-bandas de frequência, aplicada sobre sinais
não-estacionários.\cite{clark2009sum} \cite{li2005properties}. Essa técnica
assume que um sinal $x[n]$ de valores reais é representado pela soma dos
produtos entre moduladora e portadora que compõem cada sub-banda $x_k[n]$:
\begin{equation}
    x[n] = \sum_{k = 0}^{K - 1} x_k[n] = \sum_{k = 0}^{K - 1} m_k[n] c_k[n]
\end{equation}


A implementação genérica da filtragem de modulação está representada no diagrama de blocos abaixo.

\begin{figure}[h]
    \centering
    \begin{tikzpicture}[auto,>=latex']
        \tikzstyle{block} = [draw, shape=rectangle, minimum height=3em, minimum
        width=3em, node distance=4cm, line width=1pt]
        %Creating Blocks and Connection Nodes
        \node at (-2.5,0) (input) {$x[n]$};
        \node [block] (filterbank) {Banco de filtros};
        \node [block, right of=filterbank] (dem) {demodulação};
        \node [block, right of=dem] (LP) {passa-baixas};
        \node [block, below of = filterbank] (sintese) {síntese};
        \node [block, below of=dem] (res) {reconstrução};
        \node [left of=sintese, node distance=2.5cm] (output) {$x'[n]$};
        %Conecting Blocks
        \begin{scope}[line width=1pt]
            \draw[->] (input) -- (filterbank);
            \draw[->, line width=1mm] (filterbank) -- node[name=subband] {$x_k[n]$} (dem);
            \draw[->, line width=1mm] (dem) -- node[name=mod] {$m_k[n]$} (LP);
            \draw[->, line width=1mm] (dem) -- node[name=carrier] {$c_k[n]$} (res);
            \draw[->, line width=1mm] (LP) |- node[name=new_mod] {$m'_k[n]$} (res);
            \draw[->, line width=1mm] (res) -- node[name=new_subband] {$x'_k[n]$} (sintese);
            \draw[->] (sintese) -- (output);
        \end{scope}
    \end{tikzpicture}
    \caption{Filtro de modulação, em diagrama de blocos. Adaptado de \cite{li2005properties}.}
\end{figure}

O sinal banda-larga $x[n]$ passa por um banco de filtros, no
qual as  $x_k[n]$ saídas do banco de filtros passam por um módulo de
demodulação, no qual são separadas as parcelas moduladoras $m_k[n]$ e portadoras $c_k[n]$
de cada sub-banda. Cada parcela $m_k[n]$ é filtrada por um filtro passa-baixas.
Em seguida, cada sub-banda é reconstruída a partir da portadora $c_k[n]$ e da
modulação filtrada $m'_k[n]$. Finalmente, o sinal de banda larga $x'_k[n]$ é
sintetizado com as sub-bandas $x'_k[n]$ reconstruídas, obtidas a partir das
modulações filtradas.

Em aplicações de \textit{Speech Enhancement}, o projeto do banco de filtros
emula a resposta em frequência da cóclea no sistema auditivo. Por sua vez, a
frequência de corte do filtro passa-baixas é tipicamente de 16Hz, tal que
frequências abaixo desse valor estão associadas às periodicidades na construção
de frases, palavras e sílabas.