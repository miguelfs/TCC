% \chapter{Introdução}
% \section{Percepção de periodicidade}
% A audição humana é capaz de conferir qualidades subjetivas aos sons, associadas
% a parametros físicos nele presentes. Intensidade, timbre, duração e direção percebidas
% são exemplos de qualidades subjetivas, enquanto que pressão, frequência,
% espectro e envoltória são exemplos de parâmetros físicos.~\cite{rossing2002}

% A audição também é capaz de distinguir sons periódicos de sons ``aperiódicos'', ou seja,
%  sons sem periodicidade definida.

%  No contexto de sinais, um sinal é periódico se \cite{lathi2017}, para todo instante,
%  seu valor repete-se para um determinado período fundamental. Por esse motivo,
%  um sinal periódico deve começar no instante $-\infty$ e seguir indefinidamente.
%  Caso contrário, o sinal é aperiódico.


% A percepção de periodicidade no sistema auditivo possui qualidades distintas,
%  dependendo da frequência do som audível. Frequências abaixo de
% 20Hz podem ser percebidas ritmicamente, enquanto que frequências audíveis acima de
%  20Hz podem ser percebidas como \textit{Pitch}.

% \textit{Pitch} é a percepção - condicionada por outros parâmetros, como intensidade -
%  associada à altura do som, na qual o ouvinte é capaz
% de distinguir sons agudos e graves, e de ordená-los conforme sua frequência.
%  Em geral, sons com \textit{Pitch} percebido podem ser representados como a mistura de uma
% oscilação no tempo com frequência fundamental agregada às suas parciais, ou
%  harmônicos. Sons ``aperiódicos'' não possuem \textit{Pitch} definido, uma vez que as frequências das
%  componentes de sua mistura não são distribuídas harmonicamente.~\cite{langner1992}~\cite{angus2009} 

%  Um som pode ser composto por uma forma de onda periódica, que
%  repete-se em função de sua frequência fundamental f, ou período T, sendo essa 
%  frequência fundamental percebida como \textit{Pitch}. Essa definição distingue-se de um
%  trem de pulsos com determinada taxa de repetição, que não é diretamente
%  associada à sua frequência fundamental. Apesar disso, o trem de pulsos pode ser
% considerado um som periódico. Os dois casos acima demonstram que o sistema
% auditivo realiza tanto uma análise em frequência quanto uma análise no tempo
% ao perceber sons.~\cite{rossing2002} \symbl{T}{Período} \symbl{f}{Frequência}


% A percepção de periodicidade do som também pode ser conferida ao envoltória,
% definido como a variação no tempo da amplitude ou energia de uma vibração.
% Caso a envoltória oscile com uma determinada frequência, gera-se uma onda
% modulada em amplitude.
% Essa modulação pode ser percebida como uma segunda periodicidade,
% além da frequência fundamental dessa mesma onda. Dessa forma, vale destacar
% que há múltiplas dimensões temporais no estimulo acústico, e pode-se distinguir
% a ``estrutura fina'', associada ao conteúdo espectral, de seu contorno caracterizado
% pela envoltória modulado em frequência, ambos compondo a forma de onda no
% domínio do tempo.~\cite{joris2004}

% \citet{langner1992} distingue modulações lentas, abaixo de 20Hz, de
% modulações rápidas, entre 20Hz e 1000Hz. Essa distinção se dá pela percepção
% distinta nos dois casos: modulações lentas estão associadas à percepção de
% ritmo e também à taxa de sequência na construção de palavras, entre 3Hz e 4Hz,
% enquanto que modulações rápidas estão associadas a sensação desagradável
% de ``irregularidade'' (roughness) no som.

% Ao avaliar a produção de sílabas na fala, em uma variedade de línguas, dialetos,
% contextos e idades, a envoltória da forma de onda produzida expõem regularidade 
% nos intervalos de produção das sílabas, com taxas entre 2Hz e 8Hz,
% e máximos entre 4Hz e 5Hz.
% Essa 'ritimicidade' é produzida pela ação conjunta dos articuladores no trato vocal.

% is rhythmic to the degree that it has a regular temporal structure that presumably reflects deep properties of perception and production systems.
% \citet{zwicker2013} Apresentam detalhes sobre os limiares de percepção da
% Modulação de Amplitude (AM) em função das frequências e intensidades, \citet{joris2004} e
% \citet{langner1992} avaliam a resposta à modulação de cada componente do 
% sistema auditivo e nervoso, indo além do escopo do
% presente trabalho. Modulações temporais são estímulos que auxiliam na detecção,
% discriminação, identificação e localização de fontes sonoras,
% e esse papel amplo reflete-se em propriedades fisiológicas particulares a cada
% componente do sistema auditivo. \abbrev{AM}{Modulação de Amplitude}

% Falta falar: distinguir definição formal de periodicidade de quasi-periodico;
%  brevemente, como o ouvido percebe a modulação; brevemente, como modulação,
%  envoltória e formantes da fala se correlacionam; percepção de timbre de instrumentos
%   associado ao envoltória de cada frequência; inserir imagem.

% \section{Descrição do projeto}
% \subsection{Objetivos}
% \subsection{Organização do Trabalho}
% \subsection{Materiais Utilizados}

\chapter{Introdução}

\section{Justificativa}

Aplicações que dependam de captação de voz, como chamadas de voz sobre IP (VoIP,
 do inglês \textit{voice over IP}) e reconhecimento de fala, podem ter sua
 experiência de uso e assertividade melhoradas quando a fala captada é mais
 inteligível, o que pode ser potencializado com técnicas de \textit{speech
 enhancement} aplicadas sobre o áudio analisado.

No conjunto das técnicas de \textit{speech enhancement}, as que utilizam a
transformada discreta de Fourier (DFT, do inglês \textit{discrete Fourier
transform}) na forma de uma transformada rápida de fourier (FFT, do inglês
\textit{fast Fourier transform}) apresentam baixa complexidade computacional,
além de permitirem a modificação dos coeficientes do sinal no domínio da
frequência e sua ressíntese no domínio do tempo. Entretanto, o uso da DFT é mais
adequado para sinais estacionários. Sinais quase estacionários, isto é, sinais
cuja estatística é aproximadamente constante em curtos períodos, tais como a
fala —-- em janelas da ordem de milissegundos —--, podem ser bem representados
pela transformada de Fourier de curta duração (STFT, do inglês
\textit{short-time Fourier transform}). Sobre esta, é possível utilizar
algoritmos de redução de ruído, como a subtração espectral e o filtro de
Wiener~\cite{parchami2016}. Uma das técnicas evoluídas a partir da STFT é o
espectrograma de modulação —-- objeto da presente pesquisa —--, que representa
oscilações de segunda ordem no sinal de áudio analisado, no chamado domínio da
modulação.

O sinal de fala pode ser representado como a sobreposição de portadoras geradas
pelas cordas vocais, cujas amplitudes e frequências variam lentamente em
consequência das mudanças provocadas pelo trato vocal e de seus articuladores,
durante a fonação. A bibliografia demonstra que a componente AM contribui para a
inteligibilidade do sinal de fala, uma vez que a envoltória quantiza a estrutura
temporal de fonemas, sílabas e frases, atribuindo ritmicidade a essas
unidades~\cite{poeppel2020,varnet2017}. Quanto à percepção auditiva, atribui-se
à cóclea a capacidade de filtrar o som (de banda larga), em diversas sub-bandas
de banda estreita, de forma que modulações em amplitude sobre cada sub-banda
sejam passadas adiante no sistema auditivo. Portanto, o espectrograma de
modulação demonstra-se adequado para aplicações de \textit{speech enhancement},
pois permite a representação da modulação em um domínio que considere diferentes
frequências de portadora --— de forma análoga ao ``banco de filtros'' da cóclea
--— e representa, tal como a STFT, a quase estacionariedade em curtos períodos
de tempo característica em sinais de fala. Avaliações do desempenho de técnicas
de \textit{speech enhancement} que atuam no domínio do espectrograma de
modulação reforçam essa justificativa, indicando bom desempenho em índices de
inteligibilidade~\cite{schwerin2018}.

Uma segunda técnica disponível é a filtragem de modulação~\cite{atlas2005}. Essa
técnica se baseia nas evidências de que, ao aumentar gradualmente a frequência
de corte para um filtro passa-baixas aplicado sobre oscilações da envoltória de
um sinal, as componentes acima de 16Hz apresentam apenas um incremento marginal
na inteligibilidade~\cite{drullman1994}. Dessa forma, as componentes de
modulação abaixo do limiar de 16Hz são suficientes para uma boa compreensão da
fala. A partir dessa característica da percepção, o filtro de modulação é capaz
de limitar ruídos sobre a envoltória que incorrem fora da região filtrada, dessa
forma aumentando a inteligibilidade.


Por fim, a bibliografia que aborda essas técnicas, apesar de circular há cerca
de 20 anos em pesquisa, possui poucos documentos que agregam, condensam e
abordam sua evolução e suas formas mais sofisticadas, que carecem de uma
apresentação da teoria em que se baseiam ao alcance de um leitor
interessado~\cite{parchami2016,paliwal2015}.

\vspace{0.4cm}