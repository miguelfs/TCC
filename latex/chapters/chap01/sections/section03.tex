\section{Trabalhos Relacionados}
\citet{rec_sys_handbook_2022} apresentam a caracterização geral do problema
baseado em sessão, o método utilizado em alguns modelos de aprendizado de
máquina voltados a esse problema, bancos de dados apropriados para o tema e
sugestões de direções futuras.

\citet{survey_wang_2021} discutem a diferença entre \textit{Session-Based Recommender Systems}
e \textit{Sequential Recommender Systems}, apresentam a formalização matemática
do problema, características e desafios associados aos tamanhos das sessões, sua
ordenação e ao tipo das ações realizadas nas interações. Apresentam a
categorização das diferentes arquiteturas de SBRS e seus princípios básicos.

\citet{ludewig2021empirical} apresentam os resultados do comparativo entre
medidas de desempenho ao tema SBRS, com foco em redes neurais e com bancos de
dados de comércio eletrônico, músicas, etc. Conclui, a partir de pesquisa subjetiva
com usuários, que eles de fato ficaram satisfeitos com as recomendações.

Uma sequência de trabalhos
\cite{LATIFI_2021,ludewig2021empirical,ludewig_2018,ludewig_2019,sessionrec} foi
realizada a partir da biblioteca session-rec: Um \textit{framework} em linguagem
Python que agrega modelos, ferramentas de pré-processamento de dados,
treinamento e avaliação de modelos de aprendizado de máquina específicos para o
problema SBRS.

 \citet{jusbrasil2022} apresentam os resultados do comparativo de modelos SBRS em
sessões de interações de usuários com documentos jurídicos da plataforma
Jusbrasil, incluindo a recomendação vigente da plataforma. O trabalho
conclui que o SBRS supera as medidas de desempenho obtidas pela recomendação
vigente.

Uma série de livros sustenta as definições formais de de sistemas de
recomendação e de aprendizado de máquina no presente trabalho, dos autores
\citet{pml1Book}, \citet{jannach2011recommender}, \citet{ricci2010introduction},
\citet{mitchell1997} e \citet{aggarwal2016recommender}.