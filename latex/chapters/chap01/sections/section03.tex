\section{Trabalhos Relacionados}

\citet{rec_sys_handbook_2022} apresentam a caracterização geral do problema de recomendação
baseado em sessão, os modelos utilizados para a resolução desse problema, bancos de dados apropriados para o tema e
sugestões de direções futuras. \cite{rec_sys_handbook_2022, JLZ18,LATIFI_2021, quadrana2018sequence}

\citet{survey_wang_2021} discutem a diferença entre \textit{Session-Based Recommender Systems}
e \textit{Sequential Recommender Systems}, apresentam a formalização matemática
do problema, características e desafios associados aos tamanhos das sessões, sua
ordenação e o tipo das ações realizadas nas interações. Apresentam a
categorização das diferentes arquiteturas de SBRS e seus princípios básicos.

\citet{ludewig2021empirical}
apresentam os resultados do comparativo entre medidas de desempenho de
SBRS, com foco em redes neurais e com bancos de dados de comércio eletrônico,
músicas, etc. Conclui, a partir de pesquisa subjetiva com usuários, que eles de
fato ficaram satisfeitos com as recomendações.
\cite{ludewig2021empirical, ludewig_2018,ludewig_2019,ludewig2020advances}

Na bibliografia, recomendações de usuários é um problema típico de
redes sociais. \citet{cui2018dual} apresentam um conjunto de modelos e
algoritmos para recomendação de amigos a partir das preferências e relações
implícitas entre usuários. \citet{wang2014friendbook} explora os dados dos
sensores -- GPS, acelerômetros e giroscópios -- dos dispositivos móveis dos
usuários para gerar um documento representativo de seu cotidiano. As
recomendações são realizadas a partir das similaridades entre os documentos.

No contexto de SBRS aplicados à recomendações de usuários,
\citet{muvunza2023session} apresentam um modelo de SBRS para recomendação de
voluntários a organizadores. A base de dados faz parte de uma aplicação voltada
para coordenar esforços de voluntários durante a pandemia de COVID-19 na cidade de
Shenzen, China.

Vários trabalhos
\cite{LATIFI_2021,ludewig2021empirical,ludewig_2018,ludewig_2019,sessionrec,
jusbrasil2022} foram realizados a partir da biblioteca session-rec, um
\textit{framework} em linguagem Python que agrega modelos, ferramentas de
pré-processamento de dados, treinamento e avaliação de modelos de aprendizado de
máquina específicos para o problema SBRS.

 \citet{jusbrasil2022} apresentam os resultados do comparativo de modelos SBRS em
sessões de interações de usuários com documentos jurídicos da plataforma
Jusbrasil, incluindo a recomendação vigente da plataforma. O trabalho
conclui que o SBRS supera as medidas de desempenho obtidas pela recomendação
vigente.

Uma série de livros sustenta as definições formais de sistemas de
recomendação e de aprendizado de máquina no presente trabalho, tais como
\citet{pml1Book}, \citet{jannach2011recommender}, \citet{ricci2010introduction},
\citet{mitchell1997} e \citet{aggarwal2016recommender}.