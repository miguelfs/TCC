\section{Escopo e Objetivos}

O estudo limita-se ao comparativo entre modelos de SBRS aplicados a convites de
usuários no aplicativo Indaband. Mais especificamente, o modelo deve prever qual
o próximo usuário mais provável de adicionar uma faixa em determinada sessão.
Esses modelos dividem-se nos seguintes grupos: 
\begin{inparaenum}[(1)]
  \item modelos não-personalizados,
  \item modelos baseados em regras de associação,
  \item modelos baseados em vizinhos mais próximos,
  \item modelos baseados em fatoração de matrizes e
  \item redes neurais.
\end{inparaenum}

O estudo visa descobrir quais abordagens geram maior capacidade preditiva nas
recomendações de participantes, com maior cobertura e acurácia. Esses aspectos
da recomendação serão avaliados por meio de métricas objetivas apresentadas em
\ref{chap:desempenho}, tais como: \abbrev{MRR}{\textit{Mean Reciprocal Rank}}
\begin{inparaenum}[(1)]
  \item \textit{Hit rate};
  \item MRR (do inglês \textit{Mean Reciprocal Rank});
  \item Cobertura;
  \item Popularidade.
\end{inparaenum}

O objetivo é identificar, a partir das métricas citadas, se os modelos mais
robustos de redes neurais superam os modelos de base de desempenho, e quais
modelos são mais adequados para a recomendação de convites.

Também é de interesse identificar se os modelos com maior desempenho são os
mesmos da bibliografia, uma vez que o presente trabalho trata os próprios
usuários da plataforma como itens, o que é um domínio novo na bibliografia de
SBRS.
