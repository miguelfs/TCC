\section{Justificativa}

Apesar do maior interesse recente em SBRS, há questões no tema que carecem de
 pesquisa. A principal motivação para o presente trabalho é a carência de
 pesquisa e bibliografia sobre SBRS aplicados a recomendações de usuários, ou
 recomendações de grupos de usuários até o presente momento.
 
 A grande maioria dos trabalhos de SBRS apresenta uma nova arquitetura ou um
 novo modelo de aprendizado de máquina voltado à recomendação de itens, visando
 alcançar o estado da arte a partir de um \textit{dataset} público, por mais que
 existam aplicações em que uma variedade de modelos de base de comparação
 superem modelos mais arrojados.
 
 Também é de interesse da plataforma Indaband que os usuários tenham as
 ferramentas e soluções para publicarem sessões de qualidade, sem haver desgaste
 no processo de convidar e encontrar outros usuários relevantes, o que pode
 diminuir o tempo de publicação e melhorar a qualidade das sessões. Além disso,
 o conjunto de sessões com ao menos um convite aceito ou recusado tem maior taxa
 de publicação se comparadas ao conjunto de todas as sessões, como destacado na
 tabela \ref{tab:publicacao_sessoes}.

\begin{table}[htbp]
    \centering
    \begin{tabular}{|c|c|c|}
        \hline
        & \textbf{Publicadas} & \textbf{Não Publicadas} \\
        \hline
        \textbf{Todas as sessões} & 84,9\% & 15,1\% \\ 
        \hline
        \textbf{Sem convites emitidos} & 14,6\% & 85,4\% \\
        \hline
        \textbf{Com convites emitidos} & 43,2\% & 56,8\% \\ 
        \hline
        \textbf{Com convites aceitos} & 66,1\% & 33,9\% \\
        \hline
        \textbf{Com convites recusados} & 80,5\% & 19,5\% \\
        \hline
        \textbf{Com convites ignorados} & 0,6\% & 99,4\% \\
        \hline
    \end{tabular}
    \caption{Taxa de publicação de sessões nos três últimos meses de 2023.}
    \label{tab:publicacao_sessoes}
\end{table}


 Algumas das oportunidades de pesquisa em SBRS são \cite{rec_sys_handbook_2022}:
 \begin{inparaenum}[(1)]
     \item a integração com as preferências de longo prazo do usuário, em que
     trabalhos publicados sob a temática \textit{session-aware} são minoria;
     \item a inclusão dos metadados associados aos itens;
     \item a maior variedade de domínios -- por exemplo: notícias, varejo,
     música -- dos problemas representados pelos \textit{datasets}; e
     \item a inclusão do contexto atual do usuário -- região, clima, etc. --
     durante a existência de determinada sessão\end{inparaenum}.

     


