\section{Trabalhos Relacionados}

\citet{houtgast1973} descrevem a função de transferência em modulação
(MTF\abbrev{MTF}{função de transferência em modulação}, do inglês
\textit{modulation transfer function}) para ambientes fechados como forma de
medição de inteligibilidade da fala nesses locais, degradada por ruídos
quasi-estacionários ou reverberação. A dupla de autores também introduziu o
índice de transmissão de fala (STI, do inglês \textit{speech transmission
index}) \cite{houtgast1980}, uma medida mais objetiva para a capacidade de um
canal preservar a inteligibilidade da fala, cujo cálculo é efetuado a partir da
MTF.\abbrev{STI}{índice de transmissão de fala}

% A medição da MTF é realizada a partir de
% sinal de teste com envoltória bem-definida, o qual é medido na saída do canal
% com menor índice de modulação, em decorrência dos efeitos do ambiente. O valor
% da MTF é calculado pela razão das amplitudes do espectro da envoltória do sinal
% de teste e do espectro medido na saída do canal, gerando uma representação de
% índices de modulação em função da frequência das envoltórias.

% sobre cada banda oitavada, tal que cada parcela possua um peso
% associado ao seu grau de contribuição para a inteligibilidade.
% \abbrev{STI}{índice de transmissão de fala}

\citet{drullman1994}, a partir de experimento em que varia a frequência de corte
de um filtro passa-baixas sobre parcelas moduladoras em sub-bandas críticas de
sinais de fala, identificaram, entre outras evidências, que as componentes
abaixo de 16Hz são suficientes para uma boa compreensão da fala. Essas
evidências justificam a aplicabilidade da filtragem de modulação.


\citet{schimmel2007} traz um panorama da teoria e prática das ferramentas de
análise, síntese e filtragem em modulação, com argumentos sustentados sobre as
publicações de \citet{atlas2005, atlas2003} acerca de demodulação coerente,
abordada com maior profundidade por \citet{clark2012}.

% Entre as aplicações disponíveis, há o uso da densidade espectral de potência no
% domínio da modulação como \textit{features} para classificação e
% detecção de patologias nas cordas vocais~\cite{markaki2009}. 

Algumas aplicações que utilizam da análise no domínio da modulação realizam
 tarefas como~\cite{paliwal2015}: classificação \cite{markaki2009},
 reconhecimento de fala \cite{hermansky1994rasta} \cite{kanedera1999relative}
 \cite{kingsbury1998} \cite{lu2010temporal} \cite{nadeu1997filtering}
 \cite{tyagi2003mel} \cite{xiao2008normalization}.  