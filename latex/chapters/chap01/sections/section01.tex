\section{Contexto e Motivação}

Aplicações que dependam de captação de voz, como chamadas de voz sobre IP (VoIP,
 do inglês \textit{voice over IP}) e reconhecimento de fala, podem ter sua
 experiência de uso e assertividade melhoradas quando a fala captada é mais
 inteligível, o que pode ser potencializado com técnicas de \textit{speech
 enhancement} aplicadas sobre o áudio analisado.
 \abbrev{VoIP}{voz sobre IP}

No conjunto das técnicas de \textit{speech enhancement}, as que utilizam a
transformada discreta de Fourier (DFT, do inglês \textit{discrete Fourier
transform}) na forma de uma transformada rápida de Fourier (FFT, do inglês
\textit{fast Fourier transform}) apresentam baixa complexidade computacional,
além de permitirem a modificação dos coeficientes do sinal no domínio da
frequência e sua ressíntese no domínio do tempo. Entretanto, o uso da DFT é mais
adequado para sinais estacionários. Sinais quase estacionários, isto é, sinais
cuja estatística é aproximadamente constante em curtos períodos, tais como a
fala —-- em janelas da ordem de milissegundos —--, podem ser bem representados
pela transformada de Fourier de curta duração (STFT, do inglês
\textit{short-time Fourier transform}). Sobre esta, é possível utilizar
algoritmos de redução de ruído, como a subtração espectral e o filtro de
Wiener~\cite{parchami2016}. Uma das técnicas evoluídas a partir da STFT é o
espectrograma de modulação —-- objeto da presente pesquisa —--, que representa
oscilações de segunda ordem no sinal de áudio analisado, no chamado domínio da
modulação.
\abbrev{DFT}{transformada discreta de Fourier}
\abbrev{FFT}{transformada rápida de Fourier}
\abbrev{STFT}{transformada de Fourier de curta duração}

O sinal de fala pode ser representado como a sobreposição de portadoras geradas
pelas cordas vocais, cujas amplitudes e frequências variam lentamente em
consequência das mudanças provocadas pelo trato vocal e de seus articuladores,
durante a fonação. A bibliografia demonstra que a componente AM contribui para a
inteligibilidade do sinal de fala, uma vez que a envoltória quantiza a estrutura
temporal de fonemas, sílabas e frases, atribuindo ritmicidade a essas
unidades~\cite{poeppel2020,varnet2017}. Quanto à percepção auditiva, atribui-se
à cóclea a capacidade de filtrar o som (de banda larga), em diversas sub-bandas
de banda estreita, de forma que modulações em amplitude sobre cada sub-banda
sejam passadas adiante no sistema auditivo. Portanto, o espectrograma de
modulação demonstra-se adequado para aplicações de \textit{speech enhancement},
pois permite a representação da modulação em um domínio que considere diferentes
frequências de portadora --— de forma análoga ao ``banco de filtros'' da cóclea
--— e representa, tal como a STFT, a quase estacionariedade em curtos períodos
de tempo característica em sinais de fala. Avaliações do desempenho de técnicas
de \textit{speech enhancement} que atuam no domínio do espectrograma de
modulação reforçam essa justificativa, indicando bom desempenho em índices de
inteligibilidade~\cite{schwerin2018}.

Uma segunda técnica disponível é a filtragem de modulação~\cite{atlas2005}. Essa
técnica se baseia nas evidências de que, ao aumentar gradualmente a frequência
de corte para um filtro passa-baixas aplicado sobre oscilações da envoltória de
um sinal, as componentes acima de 16Hz apresentam apenas um incremento marginal
na inteligibilidade~\cite{drullman1994}. Dessa forma, as componentes de
modulação abaixo do limiar de 16Hz são suficientes para uma boa compreensão da
fala. A partir dessa característica da percepção, o filtro de modulação é capaz
de limitar ruídos sobre a envoltória que incorrem fora da região filtrada, dessa
forma aumentando a inteligibilidade.


Por fim, a bibliografia que aborda essas técnicas, apesar de circular há cerca
de 20 anos em pesquisa, possui poucos documentos que agregam, condensam e
abordam sua evolução e suas formas mais sofisticadas, que carecem de uma
apresentação da teoria em que se baseiam ao alcance de um leitor
interessado~\cite{parchami2016,paliwal2015}.

\vspace{0.4cm}