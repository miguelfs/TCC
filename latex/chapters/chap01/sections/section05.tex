\section{Materiais e Organização do Texto}

O trabalho proposto utiliza o \textit{framework} \textit{open-source}
session-rec \cite{sessionrec} escrito em Python, uma vez que contém uma ampla
variedade de modelos SBRS e métricas de avaliação, replicando o padrão de
pesquisas recentes sobre o tema. Os dados utilizados no trabalho foram cedidos
pela empresa Indaband, na condição de anonimização dos usuários tal como ordena
a Lei Geral de Proteção de Dados \cite{lgpd}. Os experimentos foram realizados
em uma máquina virtual com 12 GB de memória RAM, 75 GB de memória de
armazenamento e GPU NVIDIA Tesla T4.

O presente texto é organizado da seguinte forma: o segundo capítulo introduz a
base teórica e aborda os sistemas de recomendação de forma abrangente,
apresentando as abordagens mais tradicionais e que sustentam a intuição de
modelos mais sofisticados. Em seguida, apresenta a definição formal das
recomendações \textit{session-based} e \textit{session-aware}.

O terceiro capítulo apresenta os modelos específicos de recomendação
utilizados no comparativo, descrevendo suas arquiteturas e funcionamento.

O quarto capítulo apresenta os comparativos realizados em abordagens distintas a
partir do conjunto de dados do aplicativo Indaband, incluindo os resultados das métricas
de desempenho assim como uma análise e discussão desses resultados.

O quinto capítulo discute o desempenho dos modelos a partir das métricas
obtidas. Também apresenta a proposta de trabalhos futuros.