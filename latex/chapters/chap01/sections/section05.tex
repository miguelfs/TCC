\section{Materiais e Organização do Texto}

O trabalho proposto utiliza o \textit{framework} \textit{open-source}
session-rec \cite{sessionrec} escrito em Python, uma vez que contém uma ampla
variedade de modelos SBRS e métricas de avaliação, replicando o padrão de
pesquisas recentes sobre o tema. Os dados utilizados no trabalho foram cedidos
pela empresa Indaband, na condição de anonimização dos usuários tal como ordena
a Lei Geral de Proteção de Dados \cite{lgpd}. Os experimentos foram realizados
em uma máquina virtual com 12 GB de memória RAM, 75 GB de memória de
armazenamento e GPU NVIDIA Tesla T4.

O presente texto é organizado da seguinte forma: o segundo capítulo introduz a
base teórica e aborda diferentes sistemas de recomendação de forma abrangente,
apresentando as abordagens mais tradicionais e que sustentam a intuição de
modelos mais sofisticados do comparativo. A definição formal das recomendações
\textit{session-based} e \textit{session-aware} é apresentada, além de suas propriedades
e medidas de avaliação. Também é apresentada a base teórica de modelos de redes
neurais que constam nas arquiteturas dos sistemas de recomendação mais sofisticados,
cujas arquiteturas são descritas no mesmo capítulo.


O terceiro capítulo apresenta os comparativos realizados em abordagens
distintas. Os experimentos são feitos a partir do conjunto de dados do
aplicativo Indaband, e incluem os resultados das métricas de desempenho.


O quarto e último capítulo apresenta uma análise e avaliação dos resultados
obtidos a partir dos experimentos realizados. Também apresenta a proposta de
trabalhos futuros considerando os resultados da presente pesquisa.

