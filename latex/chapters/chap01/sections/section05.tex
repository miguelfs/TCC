\section{Materiais e Organização do Texto}

O trabalho proposto utiliza o \textit{framework} session-rec \cite{sessionrec}
em Python, uma vez que agrega os modelos SBRS e a avaliação objetiva de forma
prática, replicando o padrão de pesquisas recentes sobre o tema. 

O segundo capítulo aborda os sistemas de recomendação de forma abrangente,
apresentando abordagens baseadas em conteúdo e em filtragem
colaborativa. Em seguida, apresenta a definição formal das recomendações
\textit{session-based} e \textit{session-aware}.

O terceiro capítulo apresenta de forma mais aprofundada os diferentes modelos de
aprendizado de máquina que serão utilizados no comparativo, com suas
arquiteturas e definições formais.

O quarto capítulo apresenta o tratamento dos dados de usuários do aplicativo
Indaband, uma proposta de modelagem mais simples e outra que inclui metadados no
contexto dos itens da base de dados.

O quinto capítulo apresenta os resultados do comparativo entre os modelos de
aprendizado de máquina, com as métricas objetivas de desempenho. O capítulo
também apresenta a proposta de trabalhos futuros.