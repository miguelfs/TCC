\section{Métodos Baseados em Fatoração}

\subsection{FPMC}
O modelo Factorized Personalized Markov Chains (FPMC) é um modelo \textit{session-aware}
que utiliza cadeias de Markov para recomendações de cestas de itens. \cite{mkv}

Uma cadeia de Markov de ordem $m$ é definida como a probabilidade condicional de
um evento dados os $m$ eventos anteriores. Visando simplificar o modelo e evitar
problemas de esparsidade, o modelo FPMC utiliza uma cadeia de Markov de ordem
$1$, ou seja, a probabilidade condicional de um evento dado o evento anterior.
Dessa forma, a matriz de transição do espaço de estados é suficiente para
descrever a cadeia.

Uma vez que deseja-se recomendar cestas de itens, as transições entre estados
consistem em transições de valores binários representando cada item que descreve
determinada cesta. Para modelar as preferências de um único usuário, basta que a
matriz de transição tenha dimensões $|I| \times |I|$, onde $|I|$ é a quantidade
de itens distintos no conjunto de dados. Para modelar as preferências de todos
os usuários, é necessário compor um tensor de dimensão $|U|
\times |I| \times |I|$.

Uma vez que o tensor de transição seria esparso, a convergência de um modelo de
máxima verossimilhança seria inviável. Dessa forma, o modelo FPMC utiliza a
decomposição de Tucker para reduzir a dimensionalidade do tensor de transição. A
decomposição de Tucker é uma generalização da decomposição SVD para tensores ou
representações de ordem superior. A decomposição de Tucker consiste em decompor
um tensor em um núcleo e em matrizes de projeção para cada dimensão do tensor.

Ao aplicar o modelo FPMC para a tarefa de recomendação de itens, os parâmetros
do modelo são otimizados para a tarefa de ranqueamento de itens. Para isso, os
autores utilizam e adaptam o critério de otimização utilizado no modelo BPR
\cite{rendle2009}.

\subsection{FISM}
O modelo \textit{Factorized Item Similarity Models} (FISM) \cite{kabbur2013fism}
é um modelo que gera recomendações a partir da similaridade entre itens,
estimados a partir de representações em matrizes de fator latente. O modelo FISM
não é um modelo \textit{session-based}. Os itens avaliados por um usuário não
tem distinção de sessão, contexto ou instante de tempo, uma vez que a fatoração
de matrizes por fatores latentes não modela essas características. Cadeias de
Markov, por outro lado, são capazes de modelar características temporais em suas
transições de estado.


O modelo FISM gera as matrizes de vetores e fatores latentes a partir da matriz
de avaliações. Em vez de estimar uma avaliação particular como o produto entre
dois vetores das matrizes obtidas na fatoração, tal como descrito na equação
\ref{fator_latente}, o modelo FISM estima a avaliação como o somatório das
similaridades entre cada item avaliado pelo usuário e o item estimado. Essa
similaridade é obtida ao realizar o produto interno do vetor latente de cada
item avaliado com o fator latente do item a ser estimado.

\subsection{Fossil}
O modelo \textit{Factorized Session-based Similarity Learning} (FOSSIL) une a
abordagem baseada em similaridade do modelo FISM com a abordagem
\textit{session-aware} baseada em cadeias de Markov do modelo FPMC. O objetivo é
aproveitar a capacidade de modelar características temporais do modelo FPMC, em
razão do uso de cadeias de Markov, com a compressão sobre matrizes esparsas do
modelo FISM. Dessa forma, as preferências de longo prazo são obtidas por um
modelo de fatores latentes sobre a base completa de itens, enquanto que as
preferências de curto prazo e seus padrões sequenciais são obtidos por uma
cadeia de Markov.

\subsection{SMF}
O modelo \textit{Sequential Matrix Factorization} (SMF) \cite{ludewig_2018} assimila-se
ao modelo FOSSIL, ao combinar fatoração de matrizes com cadeias de Markov. A distinção encontra-se
na forma como a predição é calculada: o modelo substitui o vetor latente do usuário por
um vetor de preferências de sessão, que é computado como um \textit{embedding} da sessão atual.


% \cite{fuse}.