\section{Introdução}

Uma definição típica de aprendizado de máquina consiste em afirmar que ``\textit{um
programa de computador aprende com a experiência $E$ a realizar uma tarefa $T$
com medida de desempenho $P$ caso seu desempenho em $T$, medido por $P$, melhore
com a experiência $E$} ''\cite{mitchell1997}.

Por exemplo, um programa que aprende a jogar xadrez a partir de uma série de
partidas contra si mesmo, em que a vitória é a medida de desempenho. Ou ainda,
um carro autônomo que aprende a dirigir assistindo a um motorista humano, cuja
distância percorrida de forma autônoma em ambiente controlado é a medida de
desempenho.

O aprendizado supervisionado é a forma mais comum de aprendizado de máquina.
Nele, a tarefa $T$ consiste em aprender uma função $f(\mathbf{x}) = \mathbf{y}$,
de $\mathbf{x}$ entradas e $\mathbf{y}$ saídas, tipicamente sob a forma de
vetores. As entradas são chamadas de atributos ou \textit{features}, enquanto
que as saídas estão sob a forma de um ou mais rótulos ou valores-alvo. A
experiência $E$ é obtida a partir de um conjunto de tuplas de entradas e saídas
$D = \{(\mathbf{x}_n, \mathbf{y}_n)\}_{n=1}^N$, conhecido como conjunto de
treinamento, que contém $N$ amostras \cite{pml1Book}.

% In some cases an extra validation set is drawn from the training set to validate
% the performance of the learning algorithm during training; in particular
% validation sets are frequently used to determine when training should stop, in
% order to prevent overfitting. The goal is to use the training set to minimise
% some task-specific error measure E defined on the test set. For example, in a
% regression task, the usual error measure is the sum-of- squares, or squared
% Euclidean distance between the algorithm outputs and the test-set targets. For
% parametric algorithms (such as neural networks) the usual approach to error
% minimisation is to incrementally adjust the algorithm parameters to optimise a
% loss function on the training set, which is as closely related as possible to
% E. The transfer of learning from the training set to
Em alguns casos, uma parcela do conjunto de treinamento é reservada na forma de
um conjunto de validação, utilizado para validar o desempenho do algoritmo de
aprendizado durante o treinamento, determinar quando o treinamento deve parar a
fim de evitar o \textit{overfitting}, ou ainda ajustar e definir os parâmetros
que otimizam o modelo.

No aprendizado supervisionado, assume-se que cada entrada $\mathbf{x}$ no
conjunto de treinamento $D$ está associada a uma saída $y$. A tarefa $T$
trata-se de obter um mapeamento ótimo entre as entradas e as saídas. No
aprendizado não supervisionado, não há saídas determinadas à priori. As entradas
observadas não possuem saídas correspondentes \cite{pml1Book}.

% falar de aprendizado não supervisionado

% falar sobre aprendizado profundo brevemente