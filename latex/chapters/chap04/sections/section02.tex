\section{\textit{Speech Enhancement} no Domínio Acústico}
\subsection{Subtração Espectral no Domínio Acústico}

A subtração espectral no domínio acústico, em sua forma mais simples, estima que
o espectro do sinal limpo é dado pela espectro do sinal de entrada subtraído da
estimativa do espectro do ruído. Assume-se também que a fase da parcela ruidosa
é substituível pela fase do sinal de entrada \cite{loizou}. Dessa forma,
temos:

\begin{equation} \label{chap_4:sub_magn}
    \hat{S}[l, k] =
    \begin{cases}
     [|X[l, k]| - |\hat{D}[l, k]|]e^{j\angle{X[l, k]}}, & |X[l, k]| > |\hat{D}[l, k]| \\
     0, & |X[l, k]| \leq |\hat{D}[l, k]|
    \end{cases}
\end{equation}
em que $\hat{S}[l, k]$ é a estimativa do sinal de fala desejado. Note que a
parcela de módulo $|\hat{S}[l, k]|$ na expressão seria negativa caso o módulo do
ruído fosse superior ao do sinal. Para tratar esse caso, retifica-se a parcela de
módulo obtida, tal que todos os valores negativos sejam truncados a zero.

Para descrever a subtração espectral em termos de densidade espectral de
potência, obtém-se a densidade espectral do sinal de fala a partir da equação
\eqref{chap_3:stft_sum}:
\begin{equation}
    |X[l, k]|^2 = |S[l, k]|^2 + |D[l, k]|^2 + S[l, k]\cdot D^*[l,k] + S^*[l,k] \cdot D[l,k]  
\end{equation}
\begin{equation}
    |X[l, k]|^2 = |S[l, k]|^2 + |D[l, k]|^2 + 2\operatorname{Re}(S[l, k]\cdot D^*[l,k])  
\end{equation}

A densidade espectral de potência na saída é composta pela soma das densidades
espectrais de potência do sinal de fala e do ruído, acrescidos de um termo
cruzado. O operador $\operatorname{Re}(\cdot)$ descreve a parte real de um
número complexo, enquanto o operador $*$ descreve o conjugado complexo. Uma vez
que $|D[l, k]|^2$, $S[l, k]\cdot D^*[l,k]$ e $ S^*[l,k] \cdot D[l,k]$ não são
conhecidos \textit{a priori}, esses termos são aproximados pelos valores
esperados $\mathbb{E}\{|D[l,k]|^2\}$,  $\mathbb{E}\{S[l, k]\cdot D^*[l,k]\}$ e
$\mathbb{E}\{S^*[l,k] \cdot D[l,k]\}$, sendo $\mathbb{E}\{\cdot \}$ o operador
para valor esperado. Ao assumir que as parcelas que compõem o sinal são
estacionárias, que o ruído tem média igual a zero e é descorrelacionado com o sinal
de fala, temos que:

\begin{equation}
    \mathbb{E}\{S[l, k]\cdot D^*[l,k]\} = \mathbb{E}\{S[l, k]\} \cdot \mathbb{E}\{^*D[l, k]\} = 0
\end{equation}
\begin{equation}
    \mathbb{E}\{S^*[l, k]\cdot D[l,k]\} = \mathbb{E}\{S^*[l, k]\} \cdot \mathbb{E}\{D[l, k]\} = 0
\end{equation}
\begin{equation} \label{chap_4:density_spectrum}
    |X[l, k]|^2 = |S[l, k]|^2 + |D[l, k]|^2
\end{equation}

Finalmente, ao aplicar a estimativa do ruído na equação
\eqref{chap_4:density_spectrum} e truncando valores negativos tal como na
equação \eqref{chap_4:sub_magn}, a estimativa do sinal original é descrita por:

\begin{equation}
    |\hat{S}[l, k]|^2 = |X[l, k]|^2 - |\hat{D}[l, k]|^2
\end{equation}

\begin{equation} \label{chap_4:sub_pow}
    \hat{S}[l, k] =
    \begin{cases}
        |\hat{S}[l, k]|^2 = |X[l, k]|^2 - |\hat{D}[l, k]|^2, & |X[l, k]|^2 > |\hat{D}[l, k]|^2 \\
     0, & |X[l, k]| \leq |\hat{D}[l, k]|
    \end{cases}
\end{equation}

Na prática, é importante salientar que o sinal de fala não é estacionário. Mesmo
em janelas curtas de tempo da ordem de milissegundos, assume-se que o sinal é
quasi-estacionário. Além disso, dependendo da aplicação, o ruído pode ter
correlação com o sinal de fala desejado. Esses dois fatores contribuem para que
os termos cruzados permaneçam presentes em $|\hat{S}[l, k]|^2$. Apesar disso,
por questão de simplicidade, assume-se que o valor dos termos seja igual a zero.
\todo{adicionar efeito da fase}
\todo{citar quem fala das solucoes}

Ao retificar parte dos valores obtidos, também ocorre um segundo problema: são
gerados picos espectrais na vizinhança dos valores truncados. Esses picos geram
tons na estimativa obtida do sinal de fala, que não estão presentes no sinal
original.

Um método para a subtração espectral no domínio acústico é dado por
\cite{zhang2013} \label{berouti1979enhancement}: 

\begin{equation} \label{chap_04:berouti}
    |\hat{S}[l, k]|^\gamma =
    \begin{cases}
        |\hat{S}[l, k]|^\gamma - \alpha(k)|\hat{N}[l, k]|^\gamma, &  \text{caso } |\hat{S}[l, k]|^\gamma > (\alpha(k) + \beta)|\hat{N}[l, k]|^\gamma \\
        \beta|\hat{N}[l, k]|^\gamma&  \text{caso contrário}
    \end{cases}
\end{equation}

No qual $\alpha(k) \geq 1$  é o fator de subtração, $0 < \beta \ll 1 $ é o
parâmetro de piso espectral e $\gamma$ é o fator de potência.

A partir da escolha de $\alpha(k)$, é possível reduzir a intensidade dos picos
tonais característicos da subtração espectral. O parâmetro $\beta$ suaviza os
vales na vizinhança dos picos ao aumentar a intensidade nas vizinhanças,
evitando a permanência de picos estreitos. Finalmente, $\gamma = 1$ corresponde
à subtração espectral de módulo, enquanto que $\gamma = 2$ corresponde à
subtração espectral de potência. Note que, para $\alpha(k) = 1$, $\beta = 0$ e
$\gamma = 2$, a equação \eqref{chap_04:berouti} corresponde à equação
\eqref{chap_4:sub_pow}.
\todo{descrever como escolher parametros}


\subsection{Filtro de Wiener no Domínio Acústico}