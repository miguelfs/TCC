\section{\textit{Speech Enhancement} no Domínio Acústico}
\todo{enhancement por melhoramento}
\subsection{Subtração Espectral no Domínio Acústico}

A subtração espectral no domínio acústico, em sua forma mais simples, estima que
o espectro do sinal limpo é dado pela espectro do sinal de entrada subtraído da
estimativa do espectro do ruído. Assume-se também que a fase da parcela ruidosa
é substituível pela fase do sinal de entrada \cite{loizou}. Dessa forma,
temos:

\begin{equation} \label{chap_4:sub_magn}
    \hat{S}[l, k] =
    \begin{cases}
     [|X[l, k]| - |\hat{D}[l, k]|]e^{j\angle{X[l, k]}}, & |X[l, k]| > |\hat{D}[l, k]| \\
     0, & |X[l, k]| \leq |\hat{D}[l, k]|
    \end{cases}
\end{equation}
em que $\hat{S}[l, k]$ é a estimativa do sinal de fala desejado. Note que a
parcela de módulo $|\hat{S}[l, k]|$ na expressão seria negativa caso o módulo do
ruído fosse superior ao do sinal. Para tratar esse caso, retifica-se a parcela de
módulo obtida, tal que todos os valores negativos sejam truncados a zero.

Para descrever a subtração espectral em termos de densidade espectral de
potência, obtém-se a densidade espectral do sinal de fala a partir da equação
\eqref{chap_3:stft_sum}:
\begin{equation}
    |X[l, k]|^2 = |S[l, k]|^2 + |D[l, k]|^2 + S[l, k]\cdot D^*[l,k] + S^*[l,k] \cdot D[l,k]  
\end{equation}
\begin{equation}
    |X[l, k]|^2 = |S[l, k]|^2 + |D[l, k]|^2 + 2\operatorname{Re}(S[l, k]\cdot D^*[l,k])  
\end{equation}

A densidade espectral de potência na saída é composta pela soma das densidades
espectrais de potência do sinal de fala e do ruído, acrescidos de um termo
cruzado. O operador $\operatorname{Re}(\cdot)$ descreve a parte real de um
número complexo, enquanto o operador $*$ descreve o conjugado complexo. Uma vez
que $|D[l, k]|^2$, $S[l, k]\cdot D^*[l,k]$ e $ S^*[l,k] \cdot D[l,k]$ não são
conhecidos \textit{a priori}, esses termos são aproximados pelos valores
esperados $\mathbb{E}\{|D[l,k]|^2\}$,  $\mathbb{E}\{S[l, k]\cdot D^*[l,k]\}$ e
$\mathbb{E}\{S^*[l,k] \cdot D[l,k]\}$, sendo $\mathbb{E}\{\cdot \}$ o operador
para valor esperado. Ao assumir que as parcelas que compõem o sinal são
estacionárias, que o ruído tem média igual a zero e é descorrelacionado com o sinal
de fala, temos que:

\begin{equation}
    \mathbb{E}\{S[l, k]\cdot D^*[l,k]\} = \mathbb{E}\{S[l, k]\} \cdot \mathbb{E}\{^*D[l, k]\} = 0
\end{equation}
\begin{equation}
    \mathbb{E}\{S^*[l, k]\cdot D[l,k]\} = \mathbb{E}\{S^*[l, k]\} \cdot \mathbb{E}\{D[l, k]\} = 0
\end{equation}
\begin{equation} \label{chap_4:density_spectrum}
    |X[l, k]|^2 = |S[l, k]|^2 + |D[l, k]|^2
\end{equation}

Finalmente, ao aplicar a estimativa do ruído na equação
\eqref{chap_4:density_spectrum} e truncando valores negativos tal como na
equação \eqref{chap_4:sub_magn}, a estimativa do sinal original é descrita por:
% \begin{equation}
%     |\hat{S}[l, k]|^2 = |X[l, k]|^2 - |\hat{D}[l, k]|^2
% \end{equation}

\begin{equation} \label{chap_4:sub_pow}
    |\hat{S}[l, k]|^2 =
    \begin{cases}
     |X[l, k]|^2 - |\hat{D}[l, k]|^2, & \text{se }|X[l, k]|^2 > |\hat{D}[l, k]|^2 \\
     0, & \text{caso contrário.}
    \end{cases}
\end{equation}

Na prática, é importante salientar que o sinal de fala não é estacionário. Mesmo
em janelas curtas de tempo da ordem de milissegundos, assume-se que o sinal é
quasi-estacionário. Além disso, dependendo da aplicação, o ruído pode ter
correlação com o sinal de fala desejado. Esses dois fatores contribuem para que
os termos cruzados permaneçam presentes em $|\hat{S}[l, k]|^2$. Apesar disso,
por questão de simplicidade, assume-se que o valor dos termos seja igual a zero.
\todo{adicionar efeito da fase}
\todo{citar quem fala das solucoes}

Ao retificar parte dos valores obtidos, também ocorre um segundo problema: são
gerados picos espectrais na vizinhança dos valores truncados. Esses picos geram
tons na estimativa obtida do sinal de fala, que não estão presentes no sinal
original.

Um método para a subtração espectral no domínio acústico é dado por
\cite{zhang2013} \label{berouti1979enhancement}: 

\begin{equation} \label{chap_04:berouti}
    |\hat{S}[l, k]|^\gamma =
    \begin{cases}
        |X[l, k]|^\gamma - \alpha|\hat{D}[l, k]|^\gamma, &  \text{caso } |X[l, k]|^\gamma > (\alpha + \beta)|\hat{D}[l, k]|^\gamma \\
        \beta|\hat{D}[l, k]|^\gamma&  \text{caso contrário}
    \end{cases}
\end{equation}

No qual $\alpha(k) \geq 1$  é o fator de subtração, $0 < \beta \ll 1 $ é o
parâmetro de piso espectral\cite{berouti1979enhancement} e $\gamma$ é o fator de potência.

A partir da escolha de $\alpha$, é possível reduzir a intensidade dos picos
tonais característicos da subtração espectral. O parâmetro $\beta$ suaviza os
vales na vizinhança dos picos ao aumentar a intensidade nas vizinhanças,
evitando a permanência de picos estreitos.\todo{falar o q acontece com $\beta$ alto ou baixo, $\alpha$ alto ou baixo}
$\gamma = 1$ corresponde
à subtração espectral de módulo, enquanto que $\gamma = 2$ corresponde à
subtração espectral de potência. Note que, para $\alpha = 1$, $\beta = 0$ e
$\gamma = 2$, a equação \eqref{chap_04:berouti} é idêntica à equação
\eqref{chap_4:sub_pow}.


Uma vez que $\alpha$ deve ser pequeno para \textit{frames} com razão sinal-ruído
baixas e vice-versa, temos que:

\begin{equation}
    \alpha = 
    \begin{cases}
        5, & \operatorname{SNR} < -5\operatorname{dB} \\
        \alpha_0 - \frac{\operatorname{SNR}}{s}, & -5\operatorname{dB} \leq \operatorname{SNR} \leq 20\operatorname{dB}\\
        1, & \operatorname{SNR} \geq 20\operatorname{dB}
    \end{cases}
\end{equation}
no qual $\alpha_0$ é o valor de $\alpha$ para $\operatorname{SNR} =
0\operatorname{dB}$, $\frac{1}{s}$ é a inclinação da reta entre $\alpha_0$ em
$0\operatorname{dB}$ e $\alpha = 1$ em $20\operatorname{dB}$. \todo{descrever
como escolher parametros}. Para a estimativa da parcela de ruído do sinal, é
necessário identificar em quais instantes há ausência de fala a partir de um
detector de atividade vocal (VAD, do inglês \textit{voice activity detector}).


\subsection{Filtro de Wiener no Domínio Acústico}

O filtro de Wiener é o filtro ótimo e linear com erro mínimo entre o sinal
desejado e sua estimativa. Seu projeto pode ser tanto de resposta ao impulso de
duração finita (FIR, do inglês \textit{finite-duration impulse response}) quanto
de resposta ao impulso de duração infinita (IIR, do inglês
\textit{infinite-duration impulse response})\cite{loizou}.

Para um projeto de filtro FIR no domínio do tempo discreto, a estimativa do
sinal desejado é obtida ao multiplicar cada coeficiente à amostra correspondente
na saída do sinal: 

\begin{eqnarray}
    \hat{d}[n] &=& \sum_{k=0}^{M-1}h_k x[n-k]\\
    \hat{d}[n] &=& \textbf{h}^\top \textbf{x}
    \end{eqnarray}
em que $M$ é a quantidade de coeficientes, $h_k$ é o k-ésimo coeficiente do
filtro FIR, $\textbf{h}$ é o vetor de coeficientes, $(\cdot) ^\top$ é o operador
de matriz transposta e $\mathbf{x}$ é o vetor com últimas $M$ amostras na saída.
O erro ao estimar o sinal desejado é descrito por:
\begin{eqnarray}
  \label{chap_04:error}  e[n] &=& d[n] - \hat{d}[n]\\
    e[n] &=& d[n] - \textbf{h}^\top \textbf{y}
\end{eqnarray}


Por sua vez, no projeto do filtro de Wiener IIR no domínio do tempo discreto, a
saída $\hat{d}[n]$ depende tanto de amostras passadas quanto futuras de $x[n]$,
no qual $(*)$ denota a operação de convolução linear:
\begin{eqnarray}
    \hat{d}[n] &=& \sum_{k=-\infty}^{\infty}h_k x[n-k], \quad -\infty < n < \infty\\
    \label{chap_04:d_n}\hat{d}[n] &=& h[n] \ast x[n]
\end{eqnarray}
Ao descrever a equação \ref{chap_04:d_n} no domínio da frequência discreta,
obtemos a equação \ref{chap_04:d_k}, em que $\hat{D}[k]$, $H[k]$ e
$X[k]$ são as DFTs de $\hat{d}[n]$, $h[n]$ e $x[n]$, respectivamente:
\begin{equation} \label{chap_04:d_k}
    \hat{D}[k] = H[k] X[k]
\end{equation}

Em aplicações de \textit{Speech Enhancement}, partirmos das equações
\ref{chap_04:base} e \ref{chap_04:error}, que descrevem o sinal impregnado pela
parcela de ruído e a parcela de erro. No domínio da frequência
\begin{eqnarray}
    E[k] &=& D[k] - \hat{D}[k] \\
    E[k] &=& D[k] - H[k]X[k] \label{chap_4:eq_error}
\end{eqnarray}
Para minimizar o erro quadrático médio que torna o filtro ótimo, obtém-se a expressão do erro:
\begin{eqnarray}
    \mathbb{E}\left[\,|E^2[k]|\,\right] &=& \mathbb{E}\left[\,\left(D[k]-H[k]X[k]\,\right)^\ast \left(\,D[k]-H[k]X[k]\,\right)\,\right] \\
    \mathbb{E}\left[\,|E^2[k]|\,\right] &=& \mathbb{E}\left[\,|D^2[k]|\,\right] - H[k]S_{xd}[k] - H^\ast[k]S_{dx}[k] + |H[k]|^2S_{xx}[k]
    % = \mathbb{E}[|D^2[k]|] \mathbb{E}[D^\ast[k]-H[k]Y[k]] - H^\ast[k] \mathbb{E}[Y^\ast[k]D[k] + |H[k]|^2 \mathbb{E}[|Y[k]|^2]
\end{eqnarray}
em que $S_{xx}[k]$, $S_{dx}[k]$ $S_{xd}[k]$ são o espectro de potência de $x[n]$ e os espectros
de potência cruzados de $x[n]$ e $d[n]$, respectivamente:
\begin{eqnarray}
    S_{xx}[k] &=& \mathbb{E}\left[\,|X[k]|^2  \,\right] \\
    S_{dx}[k] &=& \mathbb{E}\left[\,X^*[k]D[k]\,\right] \\
    S_{xd}[k] &=& \mathbb{E}\left[\,D^*[k]X[k]\,\right]
\end{eqnarray}

No domínio da STFT, obtém-se os ganhos do filtro ao minimizar a equação equivalente à \eqref{chap_4:eq_error} \cite{parchami2016}:
\begin{equation}
    \hat{W}[k, l] = \operatorname*{argmin}_W \mathbb{E}[\,|D[k,l] - H[k,l]X[k,l|^2\,]
\end{equation}


% Uma implementação do filtro de Wiener no domínio acústico disponível no Matlab, com 
% \textit{overlap} de 50\% e tamanho fixo $L_w$ de janela pode ser feita da seguinte forma:

%  \begin{eqnarray}
%     S_{xr}[l, k] &=& X[l, k] R^{*}[l, k]\\
%     S_{xx}[l, k] &=& X[l, k] X^{*}[l, k]\\
%     Y[l,k] &=& \sum_{n = 0}^N\frac{S_{xr}[l, k]}{S_{xx}[l, k]} X[l, k]
%  \end{eqnarray}

% O erro ao estimar o sinal desejado é descrito por:
% \begin{eqnarray}
%     e[n] &=& d[n] - \hat{d}[n]\\
%     e[n] &=& d[n] - \textbf{h}^\top \textbf{y}
% \end{eqnarray}

% Ao assumir que $y[n]$ e $d[n]$ são processos aleatórios estacionários no sentido
% amplo, o erro quadrático médio de $e[n]$ é descrito por:
% \begin{eqnarray}
%     \mathbb{E}[e^2[n]] &=& \mathbb{E}[(d[n] - \textbf{h}^\top \textbf{y})^2]\\
%     \mathbb{E}[e^2[n]] &=& \mathbb{E}[d^2[n]] -2\textbf{h}^\top\mathbb{E}[\textbf{y}d[n]] + \textbf{h}^\top\mathbb{E}[\textbf{y}\textbf{y}^\top]\textbf{h}\\
%     \label{chap_04:quadratic}\mathbb{E}[e^2[n]] &=& r_{dd}[0] -2\textbf{h}^\top \textbf{r}_{yd} + \textbf{h}^\top \mathbf{R}_{yy} \textbf{h}
% \end{eqnarray}
% em que $\mathbf{R}_{yy}$ é a matriz de autocorrelação do sinal na entrada e
% $\textbf{r}_{yd}$ é a autocorrelação entre os sinais de entrada e desejado. A
% equação \ref{chap_04:quadratic} descreve a função quadrática do vetor
% $\mathbf{h}$ com um único mínimo global, cujo gradiente é zero
% \cite{vaseghi2008advanced}. A equação que descreve o gradiente é:

% \begin{equation}
%     \frac{\partial}{\partial{\mathbf{h}}}\mathbb{E}[e^2[n]] = -2\textbf{r}_{yd} + 2\textbf{h}^\top \mathbf{R}_{yy}
% \end{equation}
% Uma vez que o vetor gradiente de $\mathbf{h}$ é igual a zero no valor mínimo,
% temos que os coeficientes para o filtro ótimo são dados por:
% \begin{eqnarray}
%     \label{chap_04:wiener_solution} \mathbf{R}_{yy}\textbf{h} &=& \textbf{r}_{yd}\\
%     \textbf{h} &=& \mathbf{R}_{yy}^{-1}\textbf{r}_{yd}
% \end{eqnarray}
% Ao substituir a equação \ref{chap_04:wiener_solution} em
% \ref{chap_04:quadratic}, temos o valor do erro quadrático médio:
% \begin{eqnarray}
%     \mathbb{E}[e^2[n]] &=& r_{dd}[0] -\textbf{h}^\top \textbf{r}_{yd}\\
%     \mathbb{E}[e^2[n]] &=& r_{dd}[0] -\textbf{h}^\top \mathbf{R}_{yy}\textbf{h}
% \end{eqnarray}
% Ao assumir que os sinais são de média zero, o termo $\textbf{h}^\top
% \mathbf{R}_{yy}\textbf{h}$ é igual à variância do sinal de saída do filtro.
% Portanto, temos que:
% \begin{eqnarray}
%     \mathbb{E}[e^2[n]] &=& r_{dd}[0] - \sigma_{\hat{d}}^2\\
%     \sigma_{e}^2 &=& \sigma_{d}^2 - \sigma_{\hat{d}}^2.
% \end{eqnarray}

% Ao projetar um filtro de Wiener com resposta infinita ao impulso no domínio do
% tempo contínuo, a saída $\hat{d}[n]$ é obtida pela convolução do sinal de
% entrada com resposta ao impulso do filtro:
% \begin{equation}
%     \hat{d}[n] = h[n] \ast y[n]
% \end{equation}
% e no domínio da frequência:
% \todo{efeito da janela + convolucao circular}
% \begin{equation}
%     \hat{D}(\omega) = H(\omega)Y(\omega)
% \end{equation}
% Dessa forma, o erro para determinada frequência $\omega_k$ é dado por:
% \begin{eqnarray}
%     E(\omega_k) &=& D(\omega_k) - \hat{D}(\omega_k)\\
%     E(\omega_k) &=& D(\omega_k) - H(\omega_k)Y(\omega_k)
% \end{eqnarray}
% Para obter a resposta em frequência do filtro ótima, é necessário calcular qual
% $(\omega_k)$ minimiza o erro quadrático médio:
% \begin{eqnarray}
%     \mathbb{E}[e^2[n]] &=& \mathbb{E}[(d[n] - \textbf{h}^\top \textbf{y})^2]\\
% \end{eqnarray}