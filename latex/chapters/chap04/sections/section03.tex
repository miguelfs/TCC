\section{\textit{Speech Enhancement} no Domínio da Modulação}
\subsection{Subtração Espectral no Domínio da Modulação}

A subtração espectral de modulação é dada por \cite{paliwal2015}:

\begin{equation}
    |\hat{\mathcal{S}}[\ell, k, \mu]|^\gamma =
    \begin{cases}
        \Delta[\ell, k, \mu]^\frac{1}{\gamma}, &  \text{caso } \Delta[\ell, k, \mu] \geq  \beta|\hat{D}[l, k, \mu]|^\gamma \\
        (\beta|\hat{D}[\ell, k, \mu]]|^\gamma)^\frac{1}{\gamma}, &  \text{caso contrário}
    \end{cases}
\end{equation}
tal que
\begin{equation}
    \Delta[\ell, k, \mu] = |\mathcal{X}[\ell, k, \mu]|^\gamma - \alpha|\hat{D}[\ell, k, \mu]|^\gamma.
\end{equation}

No domínio da modulação, um VAD
pode ser descrito de forma binária para cada \textit{frame}:

\begin{equation}
    \Phi[\ell, k] =
    \begin{cases}
        1, & \text{se }\phi[\ell, k] \geq \theta\\
        0, & \text{caso contrário}
    \end{cases}
\end{equation}

em que $l$ é o índice do \textit{frame} no domínio acústico, $k$ é o índice do
$bin$ da frequência acústica e $\theta$ é o limiar de decisão. Por sua vez,
$\phi[\ell, k]$ descreve a razão sinal-ruído do sinal:
\todo{a eq precisa ser no dominio da modulacao? parseval n resolve?}
\begin{equation}
    \phi[\ell, k] = 10 \log_{10} \left( \frac{\sum_{l}|\mathcal{X}[\ell,k, \mu]|^2}{\sum_{l}|\hat{D}[l-1,k, \mu]|^2} \right)
\end{equation}
 em que $\hat{D}[l-1,k, \mu]$ é a estimativa do ruído para o $frame$ anterior. A
 estimativa do ruído para determinado $frame$ é atualizado durante a ausência de
 fala, a partir de seu valor para o $frame$ anterior, do espectro
 de modulação para aquele instante e de um fator de esquecimento $\lambda$:

\begin{equation}
    |\hat{D}[l, k, \mu]|^\gamma = |\hat{D}[l-1, k, \mu]|^\gamma + (1-\lambda)|\mathcal{X}[\ell, k, \mu]|^\lambda.
\end{equation}



\subsection{Filtro de Wiener no Domínio da Modulação}