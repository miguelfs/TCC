% \section{Modelagem com inclusão de instrumento e gênero}
\subsection{Remoção de \textit{forks} na base de treinamento} 

Uma característica particular das sessões do Indaband é a capacidade de criar
uma sessão a partir de outra já existente, modificá-la, adicionar novas
gravações e publicá-la como uma nova iteração de uma sessão já existente.

Essa funcionalidade é muito utilizada pelos usuários. A tabela \ref{tab_sessoes}
mostra que 76\% das faixas criadas são geradas via \textit{fork}. Dessa forma,
usuários distintos podem contribuir para uma mesma sessão em momentos distintos,
ou um usuário pode contribuir para uma mesma sessão de outro usuário que
era desconhecido até então.

Uma vez que o objetivo é recomendar usuários prováveis de contribuir gravando
uma faixa inédita para uma determinada sessão, é razoável gerar um cenário em
que o recomendador continue considerando os \textit{forks} como dados de insumo
para a recomendação \textit{sequence-aware}, porém limitando-se a prever ou a
recomendar apenas o cenário de adição de um conteúdo inédito.

Com essa finalidade, foi criada uma nova base de treinamento apenas com sessões
em que o último item da sequência é a adição de uma faixa inédita. Essa
identificação é feita a partir de um metadado disponível, informando se a faixa
foi gerada por um \textit{fork} ou foi criada, seja por gravação, por importação
de uma faixa ou por separação de fontes.

Para essa abordagem, é utilizado o avaliador \textit{last-item}. Esse avaliador
considera apenas o último item da sequência como o item a ser previsto, sem que
os demais itens advindos de \textit{fork} sejam recomendados.





