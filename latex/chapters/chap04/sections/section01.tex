\section{\textit{Framework} Análise-Modificação-Síntese}
 
O \textit{Framework} análise-modificação-síntese (AMS) é o procedimento que
realiza as etapas de análise com a STFT do sinal, modificação no domínio da
frequência acústica e ressíntese do sinal modificado com a STFT
inversa \cite{paliwal2015}. Caso o sinal de entrada esteja
impregnado por ruído, é possível, de posse de uma estimativa da parcela ruidosa,
subtraí-la do sinal impregnado no domínio da frequência.

Há diferentes métodos e algoritmos para a etapa de modificação no domínio
acústico, os quais também podem ser aplicados no domínio na modulação.
Geralmente, nesse procedimento, a modificação é realizada no módulo do sinal
impregnado no domínio da frequência, preservando a fase.

Assumimos um sinal impregnado por ruído: 
\begin{equation}\label{chap_04:base}
    x[n] = s[n] + d[n]
\end{equation}
em que $s[n]$ e $d[n]$ são as parcelas de sinal e de ruído, respectivamente. No
contexto de \textit{Speech Enhancement}, $s[n]$ é o sinal de fala. O objetivo
final é obter a estimativa $\hat{s}[n]$ mais próxima possível de $s[n]$.

Ao retomar a equação \ref{stft}, temos o sinal impregnado no domínio acústico
em função da STFT das parcelas $s[n]$ e $d[n]$, representadas por $S[l, k]$ e
$D[l, k]$:
\begin{equation} \label{chap_3:stft_sum}
    X[l, k] = S[l, k] + D[l, k],
\end{equation}
cuja representação em fasor é dada por:
\begin{equation}
    X[l, k] = |X[l, k]| e^{j\angle{X[l, k]}}
\end{equation}
em que $|X[l, k]|$ e $e^{j\angle{X[l, k]}}$ são módulo e fase do espectro
acústico. A subtração espectral, um dos métodos de \textit{Speech Enhancement},
estima o espectro da parcela ruidosa $d[n]$ a partir de pausas na fala contidas
em $s[n]$.

O \textit{framework} AMS para \textit{Speech Enhancement} no domínio acústico é
ilustrado no diagrama \ref{se_acoustic}. A etapa de modificação no domínio
acústico retorna $|\hat{S}[l, k]|$, que representa a estimativa do módulo do
sinal desejado. Assume-se que sua fase é igual a de $X[l, k]$. Dessa forma, a
estimativa do sinal desejado no domínio acústico é dada por $|\hat{S}[l,
k]|e^{j\angle{X[l, k]}}$. Ao efetuar a STFT inversa desse sinal, obtém-se a
estimativa $\hat{s}[n]$ do sinal desejado no domínio do tempo. O procedimento é
ilustrado no diagrama \ref{se_acoustic}.

\begin{figure}[h]
    \label{se_acoustic}
    \centering
    \begin{tikzpicture}[auto,>=latex']
        \tikzstyle{block} = [draw, shape=rectangle, minimum height=3em, minimum
        width=3em, node distance=2cm, line width=1pt]
        %Creating Blocks and Connection Nodes
        \node at (0,0) (input) {$x[n]$};
        \node [block, below of=input] (stft) {STFT no domínio acústico};
        \node [below of=stft, node distance=5em] (abs) {$|X[l, k]|$};
        \node [block, below of=abs] (s_e) {\textit{Speech Enhancement no domínio acústico}};
        \node [right of=s_e, node distance=3in] (angle) {$e^j{\angle{X[l, k]}}$};
        \node [block, below of = s_e] (estimativa) {$|\hat{S}[l, k]|e^{j\angle{X[l, k]}}$};
        \node [block, below of=estimativa] (istft) {ISTFT no domínio acústico};
        \node [below of=istft, node distance=5em] (final) {$\hat{s}[n]$};
        
        % \node [left of=sintese, node distance=2.5cm] (output) {$x'[n]$};
        %Connecting Blocks
        \begin{scope}[line width=1pt]
            \draw[->] (input) -- (stft);
            \draw[->] (stft) -- (abs);
            \draw[->] (abs) -- (s_e);
            \draw[->] (s_e) -- (estimativa);
            \draw[->] (stft) -| (angle);
            \draw[->] (angle) |- (estimativa);
            \draw[->] (estimativa) -- (istft);
            \draw[->] (istft) -- (final);
        \end{scope}
    \end{tikzpicture}
    \caption{\textit{Framework AMS para Speech Enhancement no domínio acústico}.}
\end{figure}

Por sua vez, o \textit{framework AMS} para o domínio da modulação é ilustrado no
 diagrama \ref{se_modulation}. Aplica-se a STFT no domínio da modulação ao
 módulo do espectro no domínio acústico, obtendo $|\mathcal{X}[l, k, \mu]|$ e
 $e^{j\angle{\mathcal{X}[l, k, \mu]}}$, que são o módulo e fase do espectro de
 modulação.

\begin{figure}[!ht]
    \label{se_modulation}
    \centering
    \begin{tikzpicture}[auto,>=latex']
        \tikzstyle{block} = [draw, shape=rectangle, minimum height=3em, minimum
        width=3em, node distance=2cm, line width=1pt]
        %Creating Blocks and Connection Nodes
        \node at (0,0) (input) {$x[n]$};
        \node [block, below of=input] (stft) {STFT no domínio acústico};
        \node [below of=stft, node distance=5em] (abs) {$|X[l, k]|$};
        \node [block, below of=abs] (stft_mod) {STFT no domínio da modulação};
        \node [below of=stft_mod, node distance=5em] (abs_mod) {$|\mathcal{X}[l, k, \mu]|$};
        \node [block, below of = abs_mod] (estimativa_mod) {$|\hat{\mathcal{S}}[l, k, \mu]|e^{j\angle{X[l, k]}}$};
        \node [right of=abs_mod, node distance=5cm] (angle_mod) {$e^j{\angle{\mathcal{X}[l, k, \mu]}}$};
        \node [right of=abs_mod, node distance=8cm] (angle) {$e^j{\angle{X[l, k]}}$};
        \node [block, below of=estimativa_mod] (istft_mod) {ISTFT no domínio da modulação};
        \node [block, below of = istft_mod] (estimativa) {$|\hat{S}[l, k]|e^{j\angle{X[l, k]}}$};
        \node [block, below of=estimativa] (istft) {ISTFT no domínio acústico};
        \node [below of=istft, node distance=5em] (final) {$\hat{s}[n]$};

        %Connecting Blocks
        \begin{scope}[line width=1pt]
            \draw[->] (input) -- (stft);
            \draw[->] (stft) -- (abs);
            \draw[->] (abs) -- (stft_mod);
            \draw[->] (stft_mod) -- (abs_mod);
            \draw[->] (abs_mod) -- (estimativa_mod);
            \draw[->] (estimativa_mod) -- (istft_mod);
            \draw[->] (istft_mod) -- (estimativa);
            \draw[->] (stft) -| (angle);
            \draw[->] (stft_mod) -| (angle_mod);
            \draw[->] (angle) |- (estimativa);
            \draw[->] (angle_mod) |- (estimativa_mod);
            \draw[->] (estimativa) -- (istft);
            \draw[->] (istft) -- (final);
        \end{scope}
    \end{tikzpicture}
    \caption{\textit{Framework AMS para Speech Enhancement no domínio acústico}.}
\end{figure}


% Em seguida, o espectro de modulação é obtido, tal como na equação
% \ref{modspec_stft}.

% Os sinais obtidos pela STFT nos domínios acústico e da modulação, das equações
% \ref{stft} e \ref{modspec_stft} possuem representação em módulo e fase:
% \begin{equation}
%     X[l, k] = |X[l, k]| e^{j\angle{X[l, k]}}
% \end{equation}
% \begin{equation}
%     \mathcal{X}[l, k, \mu ] = |\mathcal{X}[l, k, \mu]| e^{j\angle{\mathcal{X}[l, k, \mu]}}
% \end{equation}
% em que $|X[l, k]|$ e $e^{j\angle{X[l, k]}}$ são módulo e fase do espectro
% acústico, $|\mathcal{X}[l, k, \mu]|$ e $e^{j\angle{\mathcal{X}[l, k, \mu]}}$ são
% módulo e fase do espectro de modulação.

% Após a aplicação do método de \textit{Speech Enhancement} no domínio da
% modulação, obtém-se a o módulo da estimativa
% do sinal $s[t]$ original. Ao compor esse módulo com a fase do sinal impregnado,

% $\mathcal{Y}[l, k, m]$,
